\documentclass[10pt]{beamer}

\newcommand{\mytitle}{Adding a third normal to CLUBB}
\newcommand{\myauthor}{Sven Bergmann}
\usepackage[utf8]{inputenc}
\usepackage{graphicx}
\usetheme{CambridgeUS}
\usecolortheme{dolphin}
%\usepackage[standard, hyperref]{ntheorem}
%\usepackage{amsmath}
%\usepackage{amsfonts}
\usepackage{bm}

% set colors
\definecolor{myNewColorA}{RGB}{255,189,0}
\definecolor{myNewcolorA}{RGB}{255,189,0}
\definecolor{myNewcolorA}{RGB}{255,189,0}
\setbeamercolor{block title}{bg=myNewColorA,fg=black}
\setbeamercolor{block body}{bg=myNewColorA!20,fg=black}
\setbeamercolor{block title alerted}{bg=black, fg=myNewColorA}
\setbeamercolor{block body alerted}{bg=black!20, fg=black}

\setbeamercolor*{block title example}{bg=myNewColorA, fg = black}
\setbeamercolor*{block body example}{bg=myNewColorA!20, fg = black}
\usebeamercolor[myNewColorA]{block title alerted}
\setbeamercolor*{palette primary}{bg=myNewcolorA}
\setbeamercolor*{palette secondary}{bg=myNewcolorA, fg = white}
\setbeamercolor*{palette tertiary}{bg=myNewColorA, fg = white}
\setbeamercolor*{titlelike}{fg=myNewColorA}
\setbeamercolor*{title}{bg=myNewColorA, fg=black}
\setbeamercolor*{item}{fg=myNewColorA}
\setbeamercolor*{caption name}{fg=myNewColorA}
\usefonttheme{professionalfonts}

\hypersetup{
    pdftitle = {\mytitle},
    pdfauthor = {\myauthor}
}

\usepackage[style=alphabetic, backend=biber]{biblatex}
\addbibresource{include/bibliography.bib}

\titlegraphic{\includegraphics[height=1.5cm]{include/pictures/photo}}

\setbeamerfont{title}{size=\large}
\setbeamerfont{subtitle}{size=\small}
\setbeamerfont{author}{size=\small}
\setbeamerfont{date}{size=\small}
\setbeamerfont{institute}{size=\small}

\title[University of Wisconsin Milwaukee]{\mytitle}
\author[\myauthor]{\myauthor}
\institute[]{University of Wisconsin Milwaukee}
\date[\textcolor{white}{05/03/2024}]{05/03/2024}

\AtBeginSection[]
{
    \begin{frame}{Contents}
        \tableofcontents[currentsection]
    \end{frame}
}

\begin{document}

    \frame{\titlepage}

    \begin{frame}{Contents}
        \tableofcontents
    \end{frame}


    \section{Introduction}\label{sec:introduction}
    \begin{frame}{}

    \end{frame}


    \section{Problem}\label{sec:problem}

    \begin{frame}{}

    \end{frame}

    \subsection{Motivation}\label{subsec:motivation}

    \begin{frame}{}

    \end{frame}

    \subsection{Closing pdes}\label{subsec:closing-pdes}

    \begin{frame}{}

    \end{frame}

    \subsection{Transformations}\label{subsec:transformations}

    \begin{frame}{}

    \end{frame}

    \subsection{Goal}\label{subsec:goal}

    \begin{frame}{}

    \end{frame}

    \subsection{Inputs and Outputs}\label{subsec:inputs-and-outputs}

    \begin{frame}{}

    \end{frame}


    \section{Definitions}\label{sec:definitions}

    \subsection{Normal Distribution}\label{subsec:normal-distribution}

    \begin{frame}{Univaritate}
        We say that a random variable $X$ is distributed
        according to a normal distribution ($X \sim \mathcal{N}(\mu, \sigma^2)$) when it has the following pdf:

        \begin{definition}[pdf of a normal distribution]
            \begin{align*}
                \label{eq:pdf_normal_dist}
                f(x|\mu, \sigma^2) = \frac{1}{\sqrt{2\pi\sigma^2}}
                \exp{\left(-\frac{1}{2}\left(\frac{x-\mu}{\sigma}\right)^2\right)}.
            \end{align*}
        \end{definition}
    \end{frame}

    \begin{frame}{Multivariate}
        We say that a random vector $\bm{X}$ $(r \times r)$
        is distributed according to a multivariate normal distribution
        when it has the following joint density function~\autocite[p. 59]{izenman_modern_2008}:
        \begin{definition}[pdf of a multivariate normal distribution]
            \begin{align}
                f(\bm{x}| \bm{\mu}, \bm{\Sigma})
                = (2\pi)^{-\frac{r}{2}}
                \left|\bm{\Sigma}\right|^{-\frac{1}{2}}
                \exp\left(-\frac{1}{2}(x-\bm{\mu})^\top \bm{\Sigma}^{-1} (x-\bm{\mu})\right),
                \bm{x} \in \mathbb{R}^r,
            \end{align}
        \end{definition}
    \end{frame}

    \subsection{Moments}\label{subsec:moments}

    \begin{frame}{}

    \end{frame}


    \section{Formulas for higher-order moments}\label{sec:formulas-for-higher-order-moments}

    \begin{frame}{$\overline{w'^4}$}

    \end{frame}

    \begin{frame}{$\overline{w'^4}$}

    \end{frame}

    \begin{frame}{$\overline{w'^4}$}

    \end{frame}

    \begin{frame}{$\overline{w'^4}$}

    \end{frame}


    \section{Integration using SymPy}\label{sec:integration-using-sympy}

    \begin{frame}
        \begin{center}
            \textbf{DEMONSTRATION}
        \end{center}
    \end{frame}


    \section{Asymptotics}\label{sec:asymptotics}

    \begin{frame}{}

    \end{frame}


    \section{Summary}\label{sec:summary}

    \begin{frame}{}

    \end{frame}
    
    \section{References}
    \begin{frame}

    \end{frame}

\end{document}