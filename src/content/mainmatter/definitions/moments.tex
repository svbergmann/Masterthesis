\subsection{Moments}\label{seubsec:moments}

Especially for this thesis, we are interested in the moments of the given multivariate normal distribution.
We can express the first order moment as the mean, denoted as $\overline{X} = \mathbb{E}[X]$, where $X$ is a random variable.
The second order moment is $\mathbb{E}[X^2]$, also denoted as the variance if it is a central moment.
The standardized third and fourth order moments have special names, so called skewness and kurtosis respectively.
We denote this by the following:
\begin{align}
    \mathbb{E}[X^3]
    &= \mathbb{E}\left[\left(\frac{X-\mu}{\sigma}\right)^3\right]
    = \frac{\mu_3}{\sigma^3}
    = \frac{\mathbb{E}[(X-\mu)^3]}{(\mathbb{E}[(X-\mu)^2])^{3/2}}
    = \frac{\kappa_3}{\kappa_2^{3/2}}, \\
    \mathbb{E}[X^4]
    &= \mathbb{E}\left[\left(\frac{X-\mu}{\sigma}\right)^4\right]
    = \frac{\mathbb{E}[(X-\mu)^4]}{(\mathbb{E}[(X-\mu)^2])^2}
    = \frac{\mu_4}{\sigma^4}.
\end{align}