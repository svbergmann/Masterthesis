\section{Analytic integration}\label{sec:anaintsympy}

For simplicity and readability, we choose the check for the formula of $\overline{w'^2}$
(this is \cref{itm:checkingstep_3} from \cref{sec:steps_for_checking}).
One starts by importing and -- obviously -- installing the packages if they are not there yet.
Importing the package \mintinline{python}{display} is useful for later on printing the equations.
Thus, this results in the code in \cref{lst:import}.
\begin{listing}[!ht]
    \caption{Import statements}
    \label{lst:import}
    \begin{pythoncode}
        import sympy as sp
        from IPython.display import display
        from sympy import abc, oo, Symbol, Integral
        from sympy.stats import Normal, density
    \end{pythoncode}
\end{listing}
In this listing,
\mintinline{python}{sympy} was defined to be called \mintinline{python}{sp}
and from \mintinline{python}{sympy} we directly imported some packages,
too, which are needed later on.

Next, we define all symbols which are needed to calculate the given integral
and therefore also to print the equations nicely.
Since we are checking $\overline{w'^2}$,
we need the (self-defined) symbols listed in \cref{lst:defsymb}.
\begin{listing}[!ht]
    \caption{Defining symbols}
    \label{lst:defsymb}
    \begin{pythoncode}
        sigma_w = Symbol('\sigma_w')
        w_1 = Symbol('w_1')
        w_2 = Symbol('w_2')
        w_bar = Symbol('\overline{w}')
        sigma_lambda_w = Symbol('\sigma_{\lambda w}')
        w_prime_2_bar = Symbol('\overline{w\'^2}')
    \end{pythoncode}
\end{listing}
Having defined the symbols, we can proceed with defining the marginal distribution.
Now, we are also using \mintinline{python}{sympy.abc}
for displaying some standard symbols (\cref{lst:defmarginals}).
\begin{listing}[!ht]
    \caption{Defining the marginals}
    \label{lst:defmarginals}
    \begin{pythoncode}
        G_1_w = Normal(name='G_1_w', mean=w_1, std=sigma_w)
        G_1_w_density = density(G_1_w)(sp.abc.w)
        G_2_w = Normal(name='G_2_w', mean=w_2, std=sigma_w)
        G_2_w_density = density(G_2_w)(sp.abc.w)
        G_3_w = Normal(name='G_3_w', mean=w_bar, std=sigma_lambda_w)
        G_3_w_density = density(G_3_w)(sp.abc.w)
        G_w = ((1 - sp.abc.delta) * sp.abc.alpha * G_1_w_density +
               (1 - sp.abc.delta) * (1 - sp.abc.alpha) * G_2_w_density +
               sp.abc.delta * G_3_w_density)
    \end{pythoncode}
\end{listing}
Having done that
we can actually display the integral
which we want to compute (\cref{lst:intwp2bar}).
\begin{listing}[!ht]
    \caption{Defining and displaying the needed integral}
    \label{lst:intwp2bar}
    \begin{pythoncode}
        w_prime_2_bar_int = sp.Integral((sp.abc.w - w_bar) ** 2 * G_w, [sp.abc.w, -oo, oo])
        display(sp.Eq(w_prime_2_bar, w_prime_2_bar_int))
    \end{pythoncode}
\end{listing}
\begin{figure}[!ht]
    \centering
    \caption{Output of \cref{lst:intwp2bar}}
    \label{fig:intwp2barout}
    \begin{align}
        \overline{w'^2}
        = \int\limits_{-\infty}^{\infty}
        \left(- \overline{w} + w\right)^{2}
        \left(\frac{\sqrt{2} \delta e^{- \frac{\left(- \overline{w} + w\right)^{2}}{2 \sigma_{\lambda w}^{2}}}}{2 \sqrt{\pi} \sigma_{\lambda w}}
        + \frac{\sqrt{2} \alpha \left(1 - \delta\right) e^{- \frac{\left(w - w_{1}\right)^{2}}{2 \sigma_{w}^{2}}}}{2 \sqrt{\pi} \sigma_{w}}\right. \nonumber\\
        + \left.\frac{\sqrt{2} \cdot \left(1 - \alpha\right) \left(1 - \delta\right) e^{- \frac{\left(w - w_{2}\right)^{2}}{2 \sigma_{w}^{2}}}}{2 \sqrt{\pi} \sigma_{w}}\right)
        \, dw \nonumber
    \end{align}
\end{figure}
Looking at \cref{fig:intwp2barout},
this is exactly the integral which we want to compute.
Using the command \mintinline{python}{.doit(conds='none')} in \cref{lst:intwp2barcalc},
we can actually calculate the given integral,
where we assume that all given constants are real.
We are also using \mintinline{python}{.simplify()} here
to make the output more readable as well as more comparable to the actual function we want to check.
\begin{listing}[!ht]
    \caption{Calculating and printing the integral}
    \label{lst:intwp2barcalc}
    \begin{pythoncode}
        w_prime_2_bar_int_val = w_prime_2_bar_int.doit(conds='none').simplify()
        display(sp.Eq(w_prime_2_bar, w_prime_2_bar_int_val))
    \end{pythoncode}
\end{listing}
\begin{figure}[!ht]
    \centering
    \caption{Output of \cref{lst:intwp2barcalc}}
    \label{fig:intwp2barcalcout}
    \begin{align}
        \overline{w'^2}
        = - \overline{w}^{2} \delta + \overline{w}^{2} + 2 \overline{w} \alpha \delta w_{1} - 2 \overline{w} \alpha \delta w_{2} - 2 \overline{w} \alpha w_{1} + 2 \overline{w} \alpha w_{2} + 2 \overline{w} \delta w_{2} - 2 \overline{w} w_{2} \nonumber\\
        - \sigma_{w}^{2} \delta + \sigma_{w}^{2} + \sigma_{\lambda w}^{2} \delta - \alpha \delta w_{1}^{2} + \alpha \delta w_{2}^{2} + \alpha w_{1}^{2} - \alpha w_{2}^{2} - \delta w_{2}^{2} + w_{2}^{2} \nonumber,
    \end{align}
\end{figure}
We can now compare \cref{fig:intwp2barcalcout} to the given equation.
To do this, we first need to define the equation for \cref{eq:wp2_bar} in \cref{lst:intwp2barsym}.
\begin{listing}[!ht]
    \caption{Python function for the second order moment}
    \label{lst:intwp2barsym}
    \begin{pythoncode}
        def w_prime_2_bar_check(delta=sp.abc.delta, alpha=sp.abc.alpha, w_1=w_1, w_2=w_2,
        w_bar=w_bar, sigma_w=sigma_w, sigma_lambda_w=sigma_lambda_w):
            return (((1 - delta) * alpha * ((w_1 - w_bar) ** 2 + sigma_w ** 2)) 
                + ((1 - delta) * (1 - alpha) * ((w_2 - w_bar) ** 2 + sigma_w ** 2)) 
                + (delta * sigma_lambda_w ** 2))
    \end{pythoncode}
\end{listing}
We can print this equation using \mintinline{python}{display} again (\cref{lst:intwp2barsymprint}).
\begin{listing}[!ht]
    \caption{Printing the symbolic equation}
    \label{lst:intwp2barsymprint}
    \begin{pythoncode}
        display(sp.Eq(w_prime_2_bar, w_prime_2_bar_check()))
    \end{pythoncode}
\end{listing}
\begin{figure}[!ht]
    \centering
    \caption{Output of \cref{lst:intwp2barsymprint}}
    \label{fig:intwp2barsymprintout}
    \begin{align}
        \nonumber
        \overline{w'^2} = \sigma_{\lambda w}^{2} \delta + \alpha \left(1 - \delta\right) \left(\sigma_{w}^{2} + \left(- \overline{w} + w_{1}\right)^{2}\right) + \left(1 - \alpha\right) \left(1 - \delta\right) \left(\sigma_{w}^{2} + \left(- \overline{w} + w_{2}\right)^{2}\right)
    \end{align}
\end{figure}
The last step is to check if those two formulas are equivalent to each other.
We can do this by using \mintinline{python}{Eq(..)} from the package \mintinline{python}{SymPy}.
\mintinline{python}{factor(..)} tries to factor the given variables to make the comparison easier.
All of this can be seen in \cref{lst:intwp2barfinalcheck}.
\begin{listing}[!ht]
    \caption{Check if the integral and the given formula are the same}
    \label{lst:intwp2barfinalcheck}
    \begin{pythoncode}
        display(sp.factor(sp.Eq(w_prime_2_bar_int_val, w_prime_2_bar_check()),
            sp.abc.alpha, sp.abc.delta))
    \end{pythoncode}
\end{listing}
This code (\cref{lst:intwp2barfinalcheck}) just displays \mintinline{python}{True},
which is exactly what we wanted to have.