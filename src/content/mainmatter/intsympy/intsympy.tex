\chapter{Integration using SymPy}\label{ch:intsympy}

Throughout this thesis, symbolic manipulation plays a crucial role in verifying mathematical expressions, particularly integrals.
To achieve this, we rely on SymPy\autocite{10.7717/peerj-cs.103}, a powerful Python library for symbolic mathematics.
This chapter delves into the world of SymPy\autocite{10.7717/peerj-cs.103}, showcasing its capabilities through a detailed example.

We begin by demonstrating the analytical approach to solving an integral.

Next, we will explore the numerical side of integration.
We are going to demonstrate how SymPy\autocite{10.7717/peerj-cs.103} can be seamlessly integrated with numerical computing libraries to evaluate the integral for specific input values.
This combined approach allows us to not only verify our analytical solution but also gain valuable insights into the integral's behavior for different scenarios.

By following this step-by-step example, the reader will gain a solid understanding of how SymPy\autocite{10.7717/peerj-cs.103} can be used, not only for this thesis.

\section{Analytic integration}\label{sec:analytic-integration}

For simplicity and readability, we choose the check for the formula of $\overline{w'^2}$
(this is \cref{itm:checkingstep_3} from \cref{sec:steps-for-checking-the-formulas}).
One starts by importing and -- obviously -- installing the packages if they are not there yet.
Importing the package \mintinline{python}{display} is useful for later on printing the equations.
Thus, this results in the code in \cref{lst:import}.
\begin{listing}[!ht]
    \caption{Import statements}
    \label{lst:import}
    \begin{pythoncode}
        import sympy as sp
        from IPython.display import display
        from sympy import abc, oo, Symbol, Integral
        from sympy.stats import Normal, density
    \end{pythoncode}
\end{listing}
In this listing,
\mintinline{python}{sympy} was defined to be called \mintinline{python}{sp}
and from \mintinline{python}{sympy} we directly imported some packages,
too, which are needed later on.

Next, we define all symbols which are needed to calculate the given integral
and therefore also to print the equations nicely.
Since we are checking $\overline{w'^2}$,
we need the (self-defined) symbols listed in \cref{lst:defsymb}.
\begin{listing}[!ht]
    \caption{Defining symbols}
    \label{lst:defsymb}
    \begin{pythoncode}
        sigma_w = Symbol('\sigma_w')
        w_1 = Symbol('w_1')
        w_2 = Symbol('w_2')
        w_bar = Symbol('\overline{w}')
        sigma_lambda_w = Symbol('\sigma_{\lambda w}')
        w_prime_2_bar = Symbol('\overline{w\'^2}')
    \end{pythoncode}
\end{listing}
Having defined the symbols, we can proceed with defining the marginal distribution.
Now, we are also using \mintinline{python}{sympy.abc}
for displaying some standard symbols (\cref{lst:defmarginals}).
\begin{listing}[!ht]
    \caption{Defining the marginals}
    \label{lst:defmarginals}
    \begin{pythoncode}
        G_1_w = Normal(name='G_1_w', mean=w_1, std=sigma_w)
        G_1_w_density = density(G_1_w)(sp.abc.w)
        G_2_w = Normal(name='G_2_w', mean=w_2, std=sigma_w)
        G_2_w_density = density(G_2_w)(sp.abc.w)
        G_3_w = Normal(name='G_3_w', mean=w_bar, std=sigma_lambda_w)
        G_3_w_density = density(G_3_w)(sp.abc.w)
        G_w = ((1 - sp.abc.delta) * sp.abc.alpha * G_1_w_density +
               (1 - sp.abc.delta) * (1 - sp.abc.alpha) * G_2_w_density +
               sp.abc.delta * G_3_w_density)
    \end{pythoncode}
\end{listing}
Having done that
we can actually display the integral
which we want to compute (\cref{lst:intwp2bar}).
\begin{listing}[!ht]
    \caption{Defining and displaying the needed integral}
    \label{lst:intwp2bar}
    \begin{pythoncode}
        w_prime_2_bar_int = sp.Integral((sp.abc.w - w_bar) ** 2 * G_w, [sp.abc.w, -oo, oo])
        display(sp.Eq(w_prime_2_bar, w_prime_2_bar_int))
    \end{pythoncode}
\end{listing}
\begin{figure}[!ht]
    \centering
    \caption{Output of \cref{lst:intwp2bar}}
    \label{fig:intwp2barout}
    \begin{align}
        \overline{w'^2}
        = \int\limits_{-\infty}^{\infty}
        \left(- \overline{w} + w\right)^{2}
        \left(\frac{\sqrt{2} \delta e^{- \frac{\left(- \overline{w} + w\right)^{2}}{2 \sigma_{\lambda w}^{2}}}}{2 \sqrt{\pi} \sigma_{\lambda w}}
        + \frac{\sqrt{2} \alpha \left(1 - \delta\right) e^{- \frac{\left(w - w_{1}\right)^{2}}{2 \sigma_{w}^{2}}}}{2 \sqrt{\pi} \sigma_{w}}\right. \nonumber\\
        + \left.\frac{\sqrt{2} \cdot \left(1 - \alpha\right) \left(1 - \delta\right) e^{- \frac{\left(w - w_{2}\right)^{2}}{2 \sigma_{w}^{2}}}}{2 \sqrt{\pi} \sigma_{w}}\right)
        \, dw \nonumber
    \end{align}
\end{figure}
Looking at \cref{fig:intwp2barout},
this is exactly the integral which we want to compute.
Using the command \mintinline{python}{.doit(conds='none')} in \cref{lst:intwp2barcalc},
we can actually calculate the given integral,
where we assume that all given constants are real.
We are also using \mintinline{python}{.simplify()} here
to make the output more readable as well as more comparable to the actual function we want to check.
\begin{listing}[!ht]
    \caption{Calculating and printing the integral}
    \label{lst:intwp2barcalc}
    \begin{pythoncode}
        w_prime_2_bar_int_val = w_prime_2_bar_int.doit(conds='none').simplify()
        display(sp.Eq(w_prime_2_bar, w_prime_2_bar_int_val))
    \end{pythoncode}
\end{listing}
\begin{figure}[!ht]
    \centering
    \caption{Output of \cref{lst:intwp2barcalc}}
    \label{fig:intwp2barcalcout}
    \begin{align}
        \overline{w'^2}
        = - \overline{w}^{2} \delta + \overline{w}^{2} + 2 \overline{w} \alpha \delta w_{1} - 2 \overline{w} \alpha \delta w_{2} - 2 \overline{w} \alpha w_{1} + 2 \overline{w} \alpha w_{2} + 2 \overline{w} \delta w_{2} - 2 \overline{w} w_{2} \nonumber\\
        - \sigma_{w}^{2} \delta + \sigma_{w}^{2} + \sigma_{\lambda w}^{2} \delta - \alpha \delta w_{1}^{2} + \alpha \delta w_{2}^{2} + \alpha w_{1}^{2} - \alpha w_{2}^{2} - \delta w_{2}^{2} + w_{2}^{2} \nonumber,
    \end{align}
\end{figure}
We can now compare \cref{fig:intwp2barcalcout} to the given equation.
To do this, we first need to define the equation for \cref{eq:wp2_bar} in \cref{lst:intwp2barsym}.
\begin{listing}[!ht]
    \caption{Python function for the second order moment}
    \label{lst:intwp2barsym}
    \begin{pythoncode}
        def w_prime_2_bar_check(delta=sp.abc.delta, alpha=sp.abc.alpha, w_1=w_1, w_2=w_2,
        w_bar=w_bar, sigma_w=sigma_w, sigma_lambda_w=sigma_lambda_w):
            return (((1 - delta) * alpha * ((w_1 - w_bar) ** 2 + sigma_w ** 2)) 
                + ((1 - delta) * (1 - alpha) * ((w_2 - w_bar) ** 2 + sigma_w ** 2)) 
                + (delta * sigma_lambda_w ** 2))
    \end{pythoncode}
\end{listing}
We can print this equation using \mintinline{python}{display} again (\cref{lst:intwp2barsymprint}).
\begin{listing}[!ht]
    \caption{Printing the symbolic equation}
    \label{lst:intwp2barsymprint}
    \begin{pythoncode}
        display(sp.Eq(w_prime_2_bar, w_prime_2_bar_check()))
    \end{pythoncode}
\end{listing}
\begin{figure}[!ht]
    \centering
    \caption{Output of \cref{lst:intwp2barsymprint}}
    \label{fig:intwp2barsymprintout}
    \begin{align}
        \nonumber
        \overline{w'^2} = \sigma_{\lambda w}^{2} \delta + \alpha \left(1 - \delta\right) \left(\sigma_{w}^{2} + \left(- \overline{w} + w_{1}\right)^{2}\right) + \left(1 - \alpha\right) \left(1 - \delta\right) \left(\sigma_{w}^{2} + \left(- \overline{w} + w_{2}\right)^{2}\right)
    \end{align}
\end{figure}
The last step is to check if those two formulas are equivalent to each other.
We can do this by using \mintinline{python}{Eq(..)} from the package \mintinline{python}{SymPy}.
\mintinline{python}{factor(..)} tries to factor the given variables to make the comparison easier.
All of this can be seen in \cref{lst:intwp2barfinalcheck}.
\begin{listing}[!ht]
    \caption{Check if the integral and the given formula are the same}
    \label{lst:intwp2barfinalcheck}
    \begin{pythoncode}
        display(sp.factor(sp.Eq(w_prime_2_bar_int_val, w_prime_2_bar_check()),
            sp.abc.alpha, sp.abc.delta))
    \end{pythoncode}
\end{listing}
This code (\cref{lst:intwp2barfinalcheck}) just displays \mintinline{python}{True},
which is exactly what we wanted to have.

\section{Numeric integration}\label{sec:numintsympy}

Again, for better readability, we choose to check the formula for $\overline{w'^2\theta_l'}$
(this is \cref{itm:checkingstep_6} from \cref{sec:steps_for_checking}).
As in \cref{sec:numintsympy}, there needs to be some packages imported.
We are importing the same packages as in \cref{lst:import} together with some more (\cref{lst:importnum}).
\begin{listing}[!ht]
    \caption{Import statements}
    \label{lst:importnum}
    \begin{pythoncode}
        from itertools import product
        import pandas as pd
        import numpy as np
    \end{pythoncode}
\end{listing}
Since we are going to need some more symbols, we also need to define those.
We still use the symbols as in \cref{lst:defsymb}, together with the ones in \cref{lst:defsymbnum}.
\begin{listing}[!ht]
    \caption{Defining symbols}
    \label{lst:defsymbnum}
    \begin{pythoncode}
        sigma_lambda_theta_l = Symbol('\sigma_{\lambda\\theta_l}')
        theta_l_1 = Symbol('\\theta_{l1}') 
        theta_l_2 = Symbol('\\theta_{l2}')
        theta_l_bar = Symbol('\overline{\\theta_l}')
        sigma_theta_l_1 = Symbol('\\sigma_{\\theta_{l1}}')
        sigma_theta_l_2 = Symbol('\\sigma_{\\theta_{l2}}')
        rho_w_theta_l = Symbol('\rho_{w\\theta_l}')
        w_prime_3_bar = Symbol('\overline{w\'^3}')
        w_prime_theta_l_prime_bar = Symbol('\overline{w\'\\theta\'_l}')
        w_prime_2_theta_prime_l_bar = Symbol('\overline{w\'^2\\theta\'_l}')
        sigma_tilde_w = Symbol('\Tilde{\sigma}_w')
        lambda_w_theta = Symbol('\lambda_{w\\theta}')
        lambda_w = Symbol('\lambda_w')
    \end{pythoncode}
\end{listing}

We start defining the integral by defining the marginals (\cref{lst:defmarginalsnum}).
\begin{listing}[!ht]
    \caption{Defining the marginals}
    \label{lst:defmarginalsnum}
    \begin{pythoncode}
        G_1_w_theta = Normal(name='G_1_w_theta', mean=sp.Matrix([w_1, theta_l_1]),
            std=sp.Matrix([[sigma_w ** 2, 0], [0, sigma_theta_l_1 ** 2]]))
        G_1_w_theta_density = density(G_1_w_theta)(sp.abc.w, sp.abc.theta)
        G_2_w_theta = Normal(name='G_2_w_theta', mean=sp.Matrix([w_2, theta_l_2]),
            std=sp.Matrix([[sigma_w ** 2, 0], [0, sigma_theta_l_2 ** 2]]))
        G_2_w_theta_density = density(G_2_w_theta)(sp.abc.w, sp.abc.theta)
        G_3_w_theta = Normal(name='G_3_w_theta', mean=sp.Matrix([w_bar, theta_l_bar]),
            std=sp.Matrix([[sigma_lambda_w ** 2, 
                rho_w_theta_l * sigma_lambda_w * sigma_lambda_theta_l], 
                [rho_w_theta_l * sigma_lambda_w * sigma_lambda_theta_l, 
                sigma_lambda_theta_l ** 2]]))
        G_3_w_theta_density = sp.simplify(density(G_3_w_theta)(sp.abc.w, sp.abc.theta))
        G_w_theta = (
        (1 - sp.abc.delta) * sp.abc.alpha * density(G_1_w_theta)(sp.abc.w, sp.abc.theta) 
        + (1 - sp.abc.delta) * (1 - sp.abc.alpha) * density(G_2_w_theta)(sp.abc.w, sp.abc.theta) 
        + sp.abc.delta * G_3_w_theta_density)
    \end{pythoncode}
\end{listing}

The integral which needs to be computed is then defined as in \cref{lst:wp2thetalbarcheck}.
\begin{listing}[!ht]
    \caption{Defining and displaying the needed integral}
    \label{lst:wp2thetalbarcheck}
    \begin{pythoncode}
        w_prime_2_theta_l_prime_bar = sp.Integral((sp.abc.w - w_bar) ** 2 * 
            (sp.abc.theta - theta_l_bar) * G_w_theta, 
            [sp.abc.w, -oo, oo], [sp.abc.theta, -oo, oo])
        display(sp.Eq(w_prime_2_theta_prime_l_bar, w_prime_2_theta_l_prime_bar))
    \end{pythoncode}
\end{listing}

Here (\cref{lst:wp2thetalbarcheck}), the output is omitted for better readability.
We do not yet compute the integral, 
because due to the complexity, 
unfortunately this is not working with \mintinline{python}{SymPy}.

Since there is still the equation to check needed, 
we proceed by defining a function for that in \cref{lst:intwp2thetalbarsym}.
\begin{listing}[!ht]
    \caption{Python function for $\overline{w'^2\theta_l}$}
    \label{lst:intwp2thetalbarsym}
    \begin{pythoncode}
        def w_prime_2_theta_l_prime_bar_check(sigma_tilde_w = sigma_tilde_w, 
            delta = sp.abc.delta, lambda_w_theta = lambda_w_theta, lambda_w = lambda_w, 
            w_prime_3_bar = w_prime_3_bar, w_prime_2_bar = w_prime_2_bar, 
            w_prime_theta_l_prime_bar = w_prime_theta_l_prime_bar):
            return ((1 / (1 - sigma_tilde_w ** 2)) *
                ((1 - delta * lambda_w_theta) / (1 - delta * lambda_w)) *
                (w_prime_3_bar / w_prime_2_bar) *
                w_prime_theta_l_prime_bar)
        display(sp.Eq(w_prime_2_theta_prime_l_bar, w_prime_2_theta_l_prime_bar_check()))
    \end{pythoncode}
\end{listing}
\begin{figure}[!ht]
    \centering
    \caption{Output of \cref{lst:intwp2thetalbarsym}}
    \label{fig:intwp2thetalbarsymout}
    \begin{align}
        \nonumber
        \overline{w'^2\theta'_l}
        &= \frac{\overline{w'\theta'_l} \cdot \overline{w'^3} \left(- \lambda_{w\theta} \delta + 1\right)}{\overline{w'^2} \cdot \left(1 - \Tilde{\sigma}_w^{2}\right) \left(- \lambda_{w} \delta + 1\right)}
    \end{align}
\end{figure}
Looking at \cref{fig:intwp2thetalbarsymout}, 
there are some other equations needed like \cref{eq:w_prime_theta_l_prime_bar}, 
\cref{eq:wp3_bar}, \cref{eq:wp2_bar}, \cref{eq:sigma_w_tilde}, and \cref{eq:lambda}.
We do not list the functions to those equations here, 
because they are defined the same way like the other equations are defined as functions.

Instead, since we cannot compute the integral analytically, 
we can create a \mintinline{python}{dataframe} 
using \mintinline{python}{pandas}\autocite{mckinney-proc-scipy-2010}.
The columns for this dataframe are going to be all the inputs we have.
To get all permutations, 
this code (\cref{lst:createdataframe}) is also using \mintinline{python}{product(..)} 
from the \mintinline{python}{itertools} package.
\begin{listing}[!ht]
    \caption{Create a dataframe and putting in arbitrary numbers}
    \label{lst:createdataframe}
    \begin{pythoncode}
        df = pd.DataFrame(product([0, 1],[-2, 2],[-1, 2],[0, 3], [Rational(1, 10)],
            [Rational(3, 10)],[Rational(4, 10)],[Rational(7, 10)],[Rational(6, 10)],
            [Rational(5, 10)],[Rational(1, 10), Rational(5, 10)],[Rational(5, 10)]),
            columns=[w_1, w_2, theta_l_1, theta_l_2, sigma_theta_l_1, sigma_theta_l_2, 
            sigma_lambda_theta_l, sigma_w, sigma_lambda_w, sp.abc.alpha, sp.abc.delta,
            rho_w_theta_l])
    \end{pythoncode}
\end{listing}

We append another column which is called \enquote{checkval} 
and lists the values for the given equation to check.
\begin{listing}[!ht]
    \caption{Attaching the \enquote{checkval} column to the dataframe}
    \label{lst:attachcheckvaltodataframe}
    \begin{pythoncode}
        df['checkval'] = (df.apply(lambda x: w_prime_2_theta_l_prime_bar_check_val.subs({
                 w_1: x[w_1], w_2: x[w_2], theta_l_1: x[theta_l_1], theta_l_2: x[theta_l_2],
                 sigma_theta_l_1: x[sigma_theta_l_1], sigma_theta_l_2: x[sigma_theta_l_2],
                 sigma_lambda_theta_l: x[sigma_lambda_theta_l], sigma_w: x[sigma_w],
                 sigma_lambda_w: x[sigma_lambda_w], sp.abc.alpha: x[sp.abc.alpha],
                 sp.abc.delta: x[sp.abc.delta], rho_w_theta_l: x[rho_w_theta_l]}), axis=1))
    \end{pythoncode}
\end{listing}
This code also uses the function defined in \cref{lst:intwp2thetalbarsym}, 
where all other equations are substituted into.
The function \mintinline{python}{df.apply(..)} is used 
to apply the function given in the parenthesis to all rows of the dataframe 
by specifying a \mintinline{python}{lambda x}, 
where \mintinline{python}{x} is corresponding to the given dataframe, \mintinline{python}{df}.
Lastly, there is also the \mintinline{python}{axis=1} parameter, 
which specifies the direction of applying the function.

Next, we are actually computing $\overline{w'^2\theta_l}$ numerically by using the quadrature method 
and applying the values of this integrals 
to a new column in the dataframe (\cref{lst:attachnuminttodataframe}).
\begin{listing}[!ht]
    \caption{Attaching the \enquote{numint} column to the dataframe}
    \label{lst:attachnuminttodataframe}
    \begin{pythoncode}
        df['numint'] = (df.apply(lambda x: Rational(w_prime_2_theta_l_prime_bar.subs({
            w_1: x[w_1],w_2: x[w_2], theta_l_1: x[theta_l_1], theta_l_2: x[theta_l_2],
            sigma_theta_l_1: x[sigma_theta_l_1], sigma_theta_l_2: x[sigma_theta_l_2],
            sigma_lambda_theta_l: x[sigma_lambda_theta_l], sigma_w: x[sigma_w],
            sigma_lambda_w: x[sigma_lambda_w], sp.abc.alpha: x[sp.abc.alpha],
            sp.abc.delta: x[sp.abc.delta], rho_w_theta_l: x[rho_w_theta_l]
        }).doit(conds='none', method='quad').evalf()), axis=1))
    \end{pythoncode}
\end{listing}
Here, we are using the integral which has been specified earlier 
and adding the parameter \mintinline{python}{method='quad'} 
to the function \mintinline{python}{.doit(..)}.
After that, \mintinline{python}{.evalf(..)} just gives the numerical value.
We try to prove that the integral value equals the function value, 
hence we are computing the error between those two columns (\cref{lst:attachdiffnumtodataframe})
and take the mean (\mintinline{python}{numpy.mean(..)} from the package NumPy\autocite{harris2020array})
of these new columns (\cref{lst:calcmeannumdiff}) to see if the error is actually numerically 0.
\begin{listing}[!ht]
    \caption{Attaching the \enquote{diffnum} column to the dataframe}
    \label{lst:attachdiffnumtodataframe}
    \begin{pythoncode}
        df['diffnum'] = abs(df['checkval'].astype(float) - df['numint'].astype(float))
    \end{pythoncode}
\end{listing}
\begin{listing}[!ht]
    \caption{Calculating the mean difference}
    \label{lst:calcmeannumdiff}
    \begin{pythoncode}
        print('The mean error between the rhs and the lhs is:', np.mean(df['diffnum']))
    \end{pythoncode}
\end{listing}
\begin{figure}[!ht]
    \centering
    \caption{Output of \cref{lst:calcmeannumdiff}}
    \label{fig:calcmeannumdiffout}
    The mean error between the rhs and the lhs is: 1.3753423344481015e-124
\end{figure}

In \cref{fig:calcmeannumdiffout}, we see that the mean error is basically 0 which we wanted.
It should be noted that based on the configuration of each individual computer,
the solutions can slightly differ due to floating point arithmetic.
