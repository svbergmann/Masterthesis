\chapter{Outlook}\label{ch:outlook}

We have seen that adding a third normal distribution right in the middle of the two defined (simplified) normal distributions does not change the \enquote{closed} formulas too much, neither it makes it too complicated.
So for \gls{CLUBB} there are new parameters, e.g. $\delta$ or $\lambda_w$, which can be chosen to \enquote{tweak} the representation of the underlying \gls{pdf} for the prognosed moments.
Ultimately one would like to fit this resulting \gls{pdf} to real data, to get better relationships, as well as thresholds for some variables.
To do this there are some approaches which unfortunately have not been discussed.
A following thesis could e.g.\ incorporate some machine learning approach to learn optimal values for certain parameters.

The methodology established in \autoref{ch:intsympy} extends far beyond the immediate application within the \gls{CLUBB} model.
This chapter serves as a blueprint for a generalizable approach to verifying and analyzing integral expressions.
Its core strength lies in the utilization of SymPy, a powerful and well-supported \gls{cas}.
SymPy's community-driven nature provides continuous development and a vast library of mathematical capabilities.
By leveraging this versatile tool, we can tackle a wide range of integral expressions, both analytically and numerically.
This approach offers several advantages:
\begin{itemize}
    \item \emph{Symbolic Verification:}
    SymPy allows us to perform symbolic manipulations, enabling the derivation of exact solutions for integrals whenever possible.
    \item \emph{Numerical Approximation:}
    For integrals that are analytically intractable, SymPy seamlessly integrates with numerical computing libraries.
    This allows us to efficiently approximate the integral's value for specific parameter choices.
    This combined approach ensures we can handle a broader range of integral expressions.
    \item \emph{Generalizability and Reusability:} The framework outlined in \cref{ch:intsympy} is not specific to the context of \gls{CLUBB}.
    By focusing on the core functionalities of SymPy, this approach can be adapted to various scientific disciplines.
\end{itemize}
Overall, the methods we developed in this thesis using SymPy are not just useful for the \gls{CLUBB} model.
These methods can be applied to many other scientific problems because they can both solve integrals exactly (symbolically) and get close answers (numerically) for a wide range of equations.
SymPy, being a powerful and widely-used tool, makes this possible.