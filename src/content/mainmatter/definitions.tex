\chapter{Definitions}\label{ch:definitions}

For better understanding of the topics covered in this thesis,
it follows a brief introduction of all formulas and terms used.


\section{Normal distribution}\label{sec:normal-distribution}

We say that a random variable $X$ is distributed
according to a normal distribution ($X \sim \mathcal{N}(\mu, \sigma^2)$) when it has the following
\gls{pdf}~\autocite[p.59]{izenman_modern_2008}:

\begin{definition}[\gls{pdf} of a normal distribution]
    \begin{align}
        \label{eq:pdf_normal_dist}
        f(x|\mu, \sigma^2) = \frac{1}{\sqrt{2\pi\sigma^2}}
        \exp{\left(-\frac{1}{2}\left(\frac{x-\mu}{\sigma}\right)^2\right)},
    \end{align}
\end{definition}

where $\mu \in (-\infty, \infty)$ is the mean and $\sigma > 0$ is the standard deviation.

\subsection{Multivariate normal distribution}\label{subsec:multivariate-normal-distribution}

We say that a random vector $\bm{X}$
is distributed according to a multivariate normal distribution
when it has the following joint density function~\autocite[p. 59]{izenman_modern_2008}:

\begin{definition}[\gls{pdf} of a multivariate normal distribution]
    \begin{align}
        f(\bm{x}| \bm{\mu}, \bm{\Sigma})
        = (2\pi)^{-\frac{r}{2}}
        \left|\bm{\Sigma}\right|^{-\frac{1}{2}}
        \exp\left(-\frac{1}{2}(\bm{x}-\bm{\mu})^\top \bm{\Sigma}^{-1} (\bm{x}-\bm{\mu})\right),
        \bm{x} \in \mathbb{R}^r,
    \end{align}
    where
    \begin{align}
        \bm{\mu} =
        \begin{pmatrix}
            \mu_1  \\
            \vdots \\
            \mu_r
        \end{pmatrix}
        \in \mathbb{R}^r
    \end{align}
    is the mean vector, and
    \begin{align}
        \bm{\Sigma} =
        \begin{pmatrix}
            \sigma_1^2                & \rho_{12}\sigma_1\sigma_2 & \rho_{13}\sigma_1\sigma_3 & \ldots & \rho_{1r}\sigma_1\sigma_r \\
            \rho_{12}\sigma_1\sigma_2 & \sigma_2^2                & \rho_{23}\sigma_2\sigma_3 & \ldots & \vdots                    \\
            \rho_{13}\sigma_1\sigma_3 & \rho_{23}\sigma_2\sigma_3 & \sigma_3^2                & \ldots & \vdots                    \\
            \vdots                    & \ldots                    & \ldots                    & \ddots & \vdots                    \\
            \rho_{1r}\sigma_1\sigma_r & \ldots                    & \ldots                    & \ldots & \sigma_r^2
        \end{pmatrix}
        \in \mathbb{R}^{r\times r}
    \end{align}
    is the (symmetric, positive definite) covariance matrix,
    and $\rho$ is the correlation between the indexed components.
    This is also often expressed as $\bm{X} \sim \mathcal{N}(\bm{\mu}, \bm{\Sigma})$,
    meaning that $\bm{X}$ (r $\times$ r random vector) is distributed
    according to a multivariate normal distribution with the given parameters.
\end{definition}

\subsection{Moments}\label{subsec:moments}

Especially for this thesis, we are interested in the moments of the given multivariate normal distribution.
We can express the first order moment as the mean,
denoted as $\overline{X} = \mathbb{E}[X]$, where $X$ is a random variable.
The second order moment is $\mathbb{E}[X^2]$, also denoted as the variance if it is a central moment.
The standardized third and fourth order moments have special names,
so-called skewness and kurtosis respectively.
We denote this by the following:
\begin{align}
    \mathbb{E}[X^3]
    &= \mathbb{E}\left[\left(\frac{X-\mu}{\sigma}\right)^3\right]
    = \frac{\mathbb{E}[(X-\mu)^3]}{(\mathbb{E}[(X-\mu)^2])^{3/2}}
    = \frac{\mu_3}{\sigma^3}, \\
    \mathbb{E}[X^4]
    &= \mathbb{E}\left[\left(\frac{X-\mu}{\sigma}\right)^4\right]
    = \frac{\mathbb{E}[(X-\mu)^4]}{(\mathbb{E}[(X-\mu)^2])^2}
    = \frac{\mu_4}{\sigma^4}.
\end{align}


\section{Variates of the pdf}\label{sec:variates-of-the-pdf}

We denote the variates of the \gls{pdf} by $w$, $r_t$, and $\theta_l$,
where $w$ is the upward wind, $r_t$ is the total water mixing ratio,
and $\theta_l$ is the liquid water potential temperature~\autocite[p. 10]{larson2022clubbsilhs}.
Variables denoted by \gls{w_prime} are defined by $w - \overline{w}$
where $\overline{w}$ is the mean of $w$ over the whole pdf.
We define \gls{r_t_prime} and \gls{theta_l_prime} in the same way.