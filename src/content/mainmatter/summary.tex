\chapter{Summary}\label{ch:summary}

This thesis explores the potential benefits of incorporating a third normal distribution into the \gls{CLUBB} model,
a framework used for parameterizing atmospheric processes.
While the initial motivation for this exploration may not be immediately clear,
\cref{sec:motivation-to-add-a-third-normal-component} delves into graphical illustrations
that reveal distinct advantages associated
with this trinormal representation for capturing specific atmospheric behaviors.

Following this initial groundwork,
\cref{sec:goal-of-this-thesis} formally establishes the central objective of this research.
That is, to demonstrate the continued validity of existing \gls{CLUBB} model formulas
within the context of the proposed trinormal distribution.
Achieving this is based on using certain transformations,
shown in \cref{sec:derivation-of-trinormal-closures-by-transformation-of-binormal-closures}.

To validate the accuracy of the transformed formulas,
a verification framework is established.
This framework defines the inputs and outputs (\cref{sec:inputs-and-outputs-of-the-verification-procedure})
associated with both the forward run (code used in the model)
and the backward run (verification direction).
This is followed by presenting a step-by-step verification procedure,
ensuring that the transformed formulas are still valid.

The foundation for this investigation is laid out in \cref{ch:definitions}.
This chapter provides a brief overview of \glspl{pdf},
including both, the univariate and the multivariate normal distribution.
Additionally, it explores the concepts of second, third, and fourth-order moments,
which play a crucial role in characterizing certain statistical properties,
but more importantly
-- at least for this work --
the shapes of the distributions.

Building upon these definitions,
\cref{sec:definition-of-the-trinormal-distribution-p_tmg}
introduces the newly proposed trinormal distribution, denoted as $P_{tmg}$.
This distribution strategically positions the third normal component in-between the two existing components
within the \gls{CLUBB} model.
Furthermore, the chapter establishes key properties associated with this new trinormal mixture \gls{pdf}.

\Cref{ch:formulas-that-define-the-shape-of-the-pdf-and-moments-in-terms-of-pdf-parameters}
details the transformation of the existing \gls{CLUBB} formulas to work with the trinormal representation.
This chapter serves as a comprehensive reference for the transformed equations employed throughout the thesis.

Following the establishment of the theoretical framework and the transformed formulas,
\cref{ch:asymptotics} investigates the asymptotic behavior of the newly derived functions.
This analysis shows how the functions behave as their arguments approach specific values.
Additionally, the chapter identifies parameter value ranges or limitations
that must be considered when working with these functions.

The final chapter -- \cref{ch:integration-using-sympy} --
delves into the actual calculation of the integrals.
This chapter explores the application of a \gls{cas} (specifically, SymPy)
for tackling both, symbolic and numerical integration tasks.

In essence,
this thesis presents a comprehensive investigation
into incorporating a trinormal distribution into the \gls{CLUBB} model.
The research demonstrates the compatibility of existing \gls{CLUBB} model formulas
with the proposed trinormal representation.
