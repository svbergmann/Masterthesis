\chapter{Introduction}\label{ch:introduction}

The \gls{CLUBB} model is a powerful tool used to simulate atmospheric behavior within climate models.
This document explores an extension to the current \gls{CLUBB} framework.

Currently, \gls{CLUBB} utilizes the sum of two normal \glspl{pdf} to represent a single atmospheric grid layer.
While effective, this approach limits the model's ability to capture the full spectrum of potential cloud layer shapes.

This work proposes an innovative solution: incorporating a third normal \gls{pdf} into the \gls{CLUBB} framework.
This addition aims to enhance the model's representational capabilities while maintaining computational efficiency and numerical stability.

To achieve this, the document delves into the details of the proposed method.
We begin by establishing a foundation with clear definitions of the relevant concepts, including normal distributions and the thermodynamic scalars crucial for atmospheric modeling (\cref{ch:definitions}).

Following this foundation, \cref{ch:problem} outlines the core problem we aim to address.
Details regarding the model's inputs, outputs, and a step-by-step approach for verifying the formulas are provided.

\Cref{ch:formulas} forms the heart of this work, presenting the actual formulas associated with the extended CLUBB model.
This section details the introduction of the third normal \gls{pdf} (\cref{sec:normaldist}), the transformation of existing equations (\cref{sec:transformationequations}), and the derivation of key moments within the model (\cref{sec:dedmoments} - \cref{sec:pdfparams}).
Additionally, \cref{sec:diag_ansatz} proposes a diagnostic approach to account for the skewness of heat and moisture, while \cref{sec:prop_closure} introduces closure relations for higher-order moments based on the newly formed mixture of three normal distributions.

To handle the mathematical integrations required by the model, \cref{ch:intsympy} explores both analytic and numerical integration techniques, potentially utilizing the SymPy library (\cref{sec:anaintsympy} \& \cref{sec:numintsympy}).

Finally, \cref{ch:asymptotics} investigates the asymptotic behavior of the extended model, providing valuable insights into its performance under various conditions.

The document concludes with an outlook in \cref{ch:outlook}, outlining potential future directions for research and exploration based on the findings presented here.
