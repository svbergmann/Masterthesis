\chapter{Introduction}\label{ch:introduction}

The \gls{CLUBB} model is a powerful tool used to simulate atmospheric behavior within climate models.
This document explores an extension to the current \gls{CLUBB} framework.
Currently, \gls{CLUBB} utilizes the sum of two normal \glspl{pdf}
to represent a single atmospheric grid layer.
While effective,
this approach limits the model's ability to capture the full spectrum of potential cloud layer shapes.
This work proposes an innovative solution:
incorporating a third normal \gls{pdf} into the \gls{CLUBB} framework.
This addition aims to enhance the model's representational capabilities
while maintaining computational efficiency and numerical stability.
To achieve this, the document dives into the details of the proposed method.

We begin by outlining the core problem we aim to address (\cref{ch:problem})
by starting with the motivation,
proceeding with a short explanation on how to close turbulence \glspl{pde} (\cref{sec:closing_pdes})
and explaining how we derive the transformation from the formulas
given by the paper~\citetitle{larson2005using}\footcite{larson2005using} (\cref{sec:transformationequations}).
After that, we define the goal of this thesis (\cref{sec:goal-of-this-thesis}),
talk about the inputs and outputs (\cref{sec:inputsandoutputs})
and provide steps for checking those formulas (\cref{sec:steps_for_checking}).

Following this motivational chapter,
we establish a foundation with clear definitions of the relevant concepts,
including normal distributions and the thermodynamic scalars
crucial for atmospheric modeling (\cref{ch:definitions}).

\Cref{ch:formulas} forms the heart of this work,
presenting the actual formulas associated with the extended \gls{CLUBB} model.
This chapter details the introduction of the third normal \gls{pdf} (\cref{sec:def_trinormal})
and the derivation of key moments within the model (\cref{sec:dedmoments} - \cref{sec:pdfparams}).
Additionally,
\cref{sec:addressing-skeweness-and-achieving-closure} proposes a diagnostic approach
to account for the skewness of heat and moisture,
while \cref{sec:prop_closure} introduces analytic closure relations
for higher-order moments based on the newly formed mixture of three normal distributions.

To handle the mathematical integrations required by the model,
\cref{ch:intsympy} explores both exact parametric
and numerical integration techniques of verifying the integrals,
utilizing the SymPy library (\cref{sec:anaintsympy} \& \cref{sec:numintsympy}).

Finally,
\cref{ch:asymptotics} investigates the asymptotic behavior of the extended model,
providing valuable insights into its performance under various conditions.

Having talked about the trinormal representation within the \gls{CLUBB} model,
\cref{ch:summary} provides a concise recap of the key findings.
This summary serves as a comprehensive overview of the essential concepts
explored throughout this thesis.

The document concludes with an outlook in \cref{ch:outlook},
outlining potential future directions for research
and exploration based on the findings presented here.
