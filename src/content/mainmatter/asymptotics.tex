\chapter{Asymptotics}\label{ch:asymptotics}

Once we defined all functions,
we see that we want certain behaviors for certain values as well as
there is a need to restrict some parameter values.

We start with the \enquote{obvious} restrictions for the \gls{pdf} parameters.
The mixture fractions $\alpha$ and $\delta$ are meant to be $\alpha \in [0,1]$ and $\delta \in [0,1)$.
But since the code tries to simplify a lot of things,
the binormal representation also does not revert back to a single normal distribution.
Therefore, we have $\alpha \in (0,1)$ due to the code.
The restriction to $\delta$ makes sense in a way
that we do not really want just the third normal to predict the whole shape.
Also, most of the formulas, e.g. \cref{eq:sk_hat_w_nondim} have a $1-\delta$ in the denominator.

In \cref{sec:derivation-of-trinormal-closures-by-transformation-of-binormal-closures},
we saw,
how the transformations between the sum of two normal distributions
and the sum of three normal distributions are working.
From those transformations, we see that we want
\begin{table}[!htb]
    \centering
    \begin{tabular}{llll}
        &
        $0 < \delta\lambda_w < 1$, &
        $0 < \delta\lambda_\theta < 1$, &
        $0 < \delta\lambda_r < 1$, \\
        $\iff$ &
        $0 < \delta\frac{\sigma_{w3}^2}{\wptwo} < 1$, &
        $0 < \delta\frac{\sigma_{\theta_l 3}^2}{\thlptwo} < 1$, &
        $0 < \delta\frac{\sigma_{r_t 3}^2}{\rtptwo} < 1$, \\
        $\iff$ &
        $0 < \delta\sigma_{w3}^2 < \wptwo$, &
        $0 < \delta\sigma_{\theta_l 3}^2 < \thlptwo$, &
        $0 < \delta\sigma_{r_t 3}^2 < \rtptwo$.
    \end{tabular}
    \label{tab:table_asymp_2}
\end{table}

That is, for instance,
that the variance of $w$ over the whole \gls{pdf} has to be strictly greater than $\delta$ times
the squared standard deviation in $w$ of the third normal distribution.

To make the resulting \glspl{pdf} realizable, it turns out~\autocite{larson2022clubbsilhs}, that we need to have
$-1 < \widehat{c}_{w \theta_l}, \widehat{c}_{w r_t}, \widehat{c}_{r_t \theta_l} < 1$.
This is for instance:
\begin{align}
    c_{w r_t}^2 &< \tswfact \left(
    \frac{(1-\delta \lambda_w)(1-\delta \lambda_r)}{(1-\delta \lambda_{w r})^2}
    \right).
\end{align}
So it might be safer to set
\begin{align}
    \label{eq:lambda_corr_ineq}
    \lambda_w , \lambda_r < \lambda_{w r}
    &\iff
    \left(\frac{\sigma_{w 3}^2}{\wptwo} < \frac{\rho_{w r_t} \sigma_{w 3} \sigma_{r_t 3}}{\wprtp}\right)
    \land
    \left(\frac{\sigma_{r_t 3}^2}{\rtptwo} < \frac{\rho_{w r_t} \sigma_{w 3} \sigma_{r_t 3}}{\wprtp}\right) \nonumber\\
    &\iff
    \left(\sigma_{w 3} \wprtp < \rho_{w r_t} \sigma_{r_t 3} \wptwo\right)
    \land
    \left(\sigma_{r_t 3} \wprtp < \rho_{w r_t} \sigma_{w 3} \rtptwo\right) \nonumber\\
\end{align}
so that the \gls{rhs} is greater and the bound is less restrictive.
If we assume $\lambda_\theta = \lambda_r$ and $\lambda_{w\theta} = \lambda_{wr}$,
then the model gains 5 new \gls{pdf} parameters:
$\delta$, $\lambda_w$, $\lambda_\theta$, $\lambda_{w \theta}$, $\lambda_{\theta r}$.

Considering that the skewness goes to zero ($Sk_w \to 0$),
we want the \gls{pdf} to revert to a single normal distribution.
Therefore, we need
\begin{align}
    \delta, \lambda_w, \lambda_r, \lambda_\theta,
    \lambda_{w r}, \lambda_{w \theta}, \lambda_{\theta r}
    \to 1.
\end{align}
In this limit, there are no third-order moments anymore, so they have to go to 0 as well.
Also, we want to have that the kurtosis is approaching 3 (value of the kurtosis of a standard normal distribution).
To ensure those points, as well as no division by zero in the code, we can define the following properties:
\begin{align}
    \label{eq:delta_propto}
    \lim_{\delta \to 1} (1-\delta) \propto |Sk_w|,
\end{align}
which means that in the limit as $\delta \to 1$,
$(1-\delta)$ should \enquote{behave as} the absolute value of the skewness of $w$,
\begin{align}
    0 < \lim_{\delta\to 1} \left(\frac{1-\delta\lambda_x}{1-\delta}\right) < \infty,
\end{align}
where $x$ means any of $w$, $r_t$, or $\theta_l$,
\begin{align}
    0 < \lim_{\delta\to 1} \left(\frac{1-\delta\lambda_x}{1-\delta\lambda_y}\right) < \infty
\end{align}
where $x$ is the same as above and $y$ means any of $w$, $r_t$, or $\theta_l$, $x \neq y$,
\begin{align}
    0 < \lim_{\delta \to 1} \left(\tsw\right)^2
    = \lim_{\delta \to 1} \left(\frac{\sigma_w^2}{\wptwo} \frac{1-\delta}{1-\delta \lambda_w}\right) < 1.
\end{align}
To ensure~\cref{eq:delta_propto}, we can use a linear \enquote{fit}, which looks like
\begin{align}
    \label{eq:lambda_fit}
    \lambda_w = \lambda_\theta = \lambda_q
    = (1 - c_1) \delta + c_1,
\end{align}
where $c_1$ is some constant.
The fit for the other $\lambda$'s is
\begin{align}
    \label{eq:lambda_xy_fit}
    \lambda_{w\theta} = \lambda_{q\theta} = \lambda_{wq}
    = (1 - c_2) \delta + c_2,
\end{align}
where again $c_2$ is some constant.
Note, that we already have a definition for the $\lambda$'s (\cref{eq:lambda}).
This definition is just for the backward run though,
because we actually have to choose $\lambda$ in the forward direction.

If we now look at the limit with the proposed fit (\cref{eq:lambda_fit}),
where $x$ is one of the three variates, we get:
\begin{align}
    \lim_{\delta \to 1} \left(\frac{1 - \delta\lambda_x}{1 - \delta}\right)
    &= \lim_{\delta \to 1} \left(\frac{1 - \delta((1 - c_1) \delta + c_1)}{1 - \delta}\right)
    = \lim_{\delta \to 1} \left(\frac{1 - ((1 - c_1) \delta^2 + c_1\delta)}{1 - \delta}\right) \\
    &= \lim_{\delta \to 1} \left(\frac{1 - \delta^2 + c_1\delta^2 - c_1\delta}{1 - \delta}\right)
    = \lim_{\delta \to 1} \left(\frac{1 - \delta^2 + c_1\delta(\delta - 1)}{1 - \delta}\right) \\
    &= \lim_{\delta \to 1} \left(\frac{1 - \delta^2}{1 - \delta}\right)
    - \lim_{\delta \to 1} \left(\frac{c_1\delta(1 - \delta)}{1 - \delta}\right) \\
    \text{(L'Hôpital)}
    &\overset{\left[\frac{0}{0}\right]}{=} \lim_{\delta \to 1} \left(\frac{-2\delta}{-1}\right)
    - \lim_{\delta \to 1} \left(c_1\delta\right) \\
    &= 2 - c_1.
\end{align}
Then, we can also define the range of $c_1$,
which should be $(0, 2)$ because we want to have $0 < \delta\lambda_w < 1$.
For the reciprocal, we then have
\begin{align}
    \lim_{\delta \to 1} \left(\frac{1 - \delta}{1 - \delta\lambda_x}\right)
    &= \lim_{\delta \to 1} \left(\frac{1 - \delta}{1 - \delta^2 + c_1\delta^2 - c_1\delta}\right)
    \overset{\left[\frac{0}{0}\right]}{=} \lim_{\delta \to 1} \left(\frac{-1}{- 2\delta + 2c_1\delta - c_1}\right) \\
    &= \frac{-1}{-2 + c_1} = \frac{1}{2 - c_1}.
\end{align}
Another limit also show up very often, which is
\begin{align}
    \lim_{\delta \to 1} \left(\frac{(1 - \delta\lambda_x)^2}{1 - \delta}\right)
    &= \lim_{\delta \to 1} \left(\frac{(1 - \delta\lambda_x) (1 - \delta\lambda_x)}{1 - \delta}\right) \\
    &= (2 - c_1) \lim_{\delta \to 1} (1 - \delta \lambda_x) \\
    &= (2 - c_1) \lim_{\delta \to 1} (1 - \delta^2 + \delta^2 c_2 - \delta c_2) = 0.
\end{align}
Since the formula for $\lambda_{w\theta} = \lambda_{q\theta} = \lambda_{wq}$ is the same just with a different constant $c_2$,
we can also calculate another limit for a fraction which appears very frequently.
Again, we use $xy$ for $w\theta_l$, $wr_t$, or $\theta_l r_t$.
\begin{align}
    \lim_{\delta \to 1} \left(\frac{1 - \delta \lambda_{xy}}{1 - \delta \lambda_x}\right)
    &= \lim_{\delta \to 1} \left(\frac{1 - \delta ((1 - c_2) \delta + c_2)}{1 - \delta ((1 - c_1) \delta + c_1)}\right)
    = \lim_{\delta \to 1} \left(\frac{1 - \delta (\delta - \delta c_2 + c_2)}{1 - \delta (\delta - \delta c_1 + c_1)}\right) \\
    &= \lim_{\delta \to 1} \left(\frac{1 - \delta^2 + \delta^2 c_2 - \delta c_2}{1 - \delta^2 + \delta^2 c_1 - \delta c_1}\right)
    \overset{\left[\frac{0}{0}\right]}{=}
    \lim_{\delta \to 1} \left(\frac{- 2\delta + 2\delta c_2 - c_2}{- 2\delta + 2\delta c_1 - c_1}\right) \\
    &= \lim_{\delta \to 1} \left(\frac{- 2\delta + 2\delta c_2 - c_2}{- 2\delta + 2\delta c_1 - c_1}\right) \\
    &= \frac{-2 + c_1}{-2 + c_2},
\end{align}
For the reciprocal it is just
\begin{align}
    \lim_{\delta \to 1} \left(\frac{1 - \delta \lambda_x}{1 - \delta \lambda_{xy}}\right) = \frac{-2 + c_2}{-2 + c_1}.
\end{align}

We now proceed and calculate all the other limits.
First, we use (based on \cref{eq:sigma_w_tilde})
\begin{align}
    \label{eq:sigma_w_tilde_2}
    \tsw^2
    &= \frac{\sigma_w^2 (1-\delta)}{\wptwo (1-\delta\lambda_w)},
\end{align}
and
\begin{align}
    \label{eq:sigma_w_tilde_4}
    \tsw^4
    &= \frac{\sigma_w^4 (1-\delta)^2}{\left(\wptwo\right)^2 \left(1-\delta\lambda_w\right)^2}.
\end{align}

We also treat the lower-order moments as fixed.
To make the calculation even more readable,
we calculate two limits here.
\begin{align}
    \label{eq:sigma_w_tilde_2_limit_delta_to_1}
    \lim_{\delta \to 1}\left(\tsw^2\right)
    = \lim_{\delta \to 1}\left(\frac{\sigma_w^2 (1-\delta)}{\wptwo (1-\delta\lambda_w)}\right)
    = \left(\frac{1}{2 - c_1}\right) \cdot \frac{\sigma_w^2}{\wptwo},
\end{align}
and
\begin{align}
    \label{eq:sigma_w_tilde_4_limit_delta_to_1}
    \lim_{\delta \to 1}\left(\tsw^4\right)
    &= \lim_{\delta \to 1}\left(\frac{\sigma_w^4 (1-\delta)^2}{\left(\wptwo\right)^2 \left(1-\delta\lambda_w\right)^2}\right)
    = \left(\frac{\sigma_w^4}{\left(\wptwo\right)^2}\right) \lim_{\delta \to 1} \left(\frac{1-\delta}{1-\delta\lambda_w}\right)^2 \\
    &= \left(\frac{1}{2 - c_1}\right)^2 \cdot \left(\frac{\sigma_w^2}{\wptwo}\right)^2.
\end{align}

We want to see how the \enquote{main} equations behave in the limit of $\delta \to 1$.
For those limits, we assume all lower-order moments as constants because they are just given to us in the code.


\section{Limit for \texorpdfstring{$\wpfour$}{wprime4bar} as \texorpdfstring{$\delta$}{delta} goes to 1}
\label{sec:limit-for-wprime4bar-as-delta-goes-to-1}

\begin{align}
    \label{eq:limit_wprime4_delta_to_1}
    \lim_{\delta \to 1}\left(\wpfour\right)
    = \left(
    \frac{
        \left(\wpthree\right)^2
    }{
        (1 - \delta\lambda_w)
        \left(\wptwo - \frac{\sigma_w^2}{2 - c_1}\right)
    }
    \right)
    + 3 \left(\wptwo\right)^2
\end{align}
Looking at this limit (for the calculation refer to \cref{sec:calculation-for-the-limit-for-wprime4bar-as-delta-goes-to-1}),
where we held the lower-order moments fixed and not depending on $\delta$,
unfortunately, we see that the fourth-order moment of $w$ diverges.
Using the additional assumption from above though,
where we assumed,
that the skewness goes as $1-\delta$,
this limit converges to $3 \left(\wptwo\right)^2$.
This is exactly as expected,
because in the limit as $\delta \to 1$ we want to revert to a single normal distribution.


\section{Limit for \texorpdfstring{$\wptwothlp$}{wprime2thetalbar} as \texorpdfstring{$\delta$}{delta} goes to 1}
\label{sec:limit-for-wprime2thetalbar-as-delta-goes-to-1}

\begin{align}
    \label{eq:limit_wprime2thetalbar_delta_to_1}
    \lim_{\delta \to 1}\left(\wptwothlp\right)
    = \frac{(c_1 - 2)^2 \wpthlp \cdot \wpthree}{(c_2 - 2)\left((c_1 - 2)\wptwo + \sigma_w^2\right)}
\end{align}
This limit (for the calculation refer to \cref{sec:calculation-for-the-limit-for-wprime2thetalprimebar-as-delta-goes-to-1})
is finite, where one just needs to pay attention on how to choose $c_1$ and $c_2$ respectively.
Also, again with taking the additional assumption from above,
the third-order-moment goes to zero as $\delta \to 1$.
This results in this limit also converging to 0,
which also makes sense.


\section{Limit for \texorpdfstring{$\wptwothlp$}{wprimethetaltwobar} as \texorpdfstring{$\delta$}{delta} goes to 1}
\label{sec:limit-for-wprimethetal2bar-as-delta-goes-to-1}

Here, looking at the calculation of the limit
(\cref{sec:calculation-for-the-limit-for-wprimethetalprime2bar-as-delta-goes-to-1}),
where no additional assumption were used at first,
we can see that the solution does not give too much insights for this limit.
If we instead again use the same assumption as above,
specifically, that the third-order-moments go to zero as $\delta$ approaches 1,
we see that this limit is also going to zero, exactly as wanted.

\begin{align}
    \label{eq:limit_wprimethetalprime2bar_delta_to_1}
    \lim_{\delta \to 1}\left(\wptwothlp\right)
    = \lim_{\delta \to 1}
    \left(
    \frac{2}{3}
    \frac{(1-\delta \lambda_{w \theta})^2}{(1-\delta \lambda_w)^2}
    \frac{1}{\tswfact^2}
    \frac{\wpthree}{\left(\wptwo \right)^2}
    \left(\wpthlp \right)^2
    +
    \frac{1}{3}
    \frac{(1-\delta \lambda_w)}{(1-\delta \lambda_{w \theta})}
    \tswfact
    \frac{\wptwo \; \thlpthree}{\wpthlp}
    \right)
\end{align}