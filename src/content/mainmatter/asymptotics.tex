\chapter{Asymptotics}\label{ch:asymptotics}

Once we defined all functions,
we see that we want certain behaviors for certain values as well as
there is a need to restrict some parameter values.

We start with the \enquote{obvious} restrictions for the \gls{pdf} parameters.
The mixture fractions $\alpha$ and $\delta$ are meant to be $\alpha \in [0,1]$ and $\delta \in [0,1)$.
But since the code tries to simplify a lot of things,
the binormal representation also does not revert back to a single normal distribution.
Therefore, we have $\alpha \in (0,1)$ due to the code.
The restriction to $\delta$ makes sense in a way
that we do not really want just the third normal to predict the whole shape.
Also, most of the formulas, e.g. \cref{eq:sk_hat_w_nondim} have a $1-\delta$ in the denominator.

In \cref{sec:derivation-of-trinormal-closures-by-transformation-of-binormal-closures},
we saw,
how the transformation between the sum of two normal distributions
and the sum of three normal distributions are working.
From those transformations, we see that we want
\begin{table}[!htb]
    \centering
    \begin{tabular}{llll}
        &
        $0 < \delta\lambda_w < 1$, &
        $0 < \delta\lambda_\theta < 1$, &
        $0 < \delta\lambda_r < 1$, \\
        $\iff$ &
        $0 < \delta\frac{\sigma_{w3}^2}{\wptwo} < 1$, &
        $0 < \delta\frac{\sigma_{\theta_l 3}^2}{\thlptwo} < 1$, &
        $0 < \delta\frac{\sigma_{r_t 3}^2}{\rtptwo} < 1$, \\
        $\iff$ &
        $0 < \delta\sigma_{w3}^2 < \wptwo$, &
        $0 < \delta\sigma_{\theta_l 3}^2 < \thlptwo$, &
        $0 < \delta\sigma_{r_t 3}^2 < \rtptwo$.
    \end{tabular}
    \label{tab:table_asymp_2}
\end{table}

That is, for instance,
that the variance of $w$ over the whole \gls{pdf} has to be strictly greater than $\delta$ times
the squared standard deviation in $w$ of the third normal distribution.

To make the resulting \glspl{pdf} realizable, it turns out\autocite{larson2022clubbsilhs}, that we need to have
$-1 < \widehat{c}_{w \theta_l}, \widehat{c}_{w r_t}, \widehat{c}_{r_t \theta_l} < 1$.
This is for instance:
\begin{align}
    c_{w r_t}^2 &< \tswfact \left(
    \frac{(1-\delta \lambda_w)(1-\delta \lambda_r)}{(1-\delta \lambda_{w r})^2}
    \right)
\end{align}
So it might be safer to set
\begin{align}
    \label{eq:lambda_corr_ineq}
    \lambda_w , \lambda_r < \lambda_{w r}
    &\iff
    \left(\frac{\sigma_{w 3}^2}{\wptwo} < \frac{\rho_{w r_t} \sigma_{w 3} \sigma_{r_t 3}}{\wprtp}\right)
    \land
    \left(\frac{\sigma_{r_t 3}^2}{\rtptwo} < \frac{\rho_{w r_t} \sigma_{w 3} \sigma_{r_t 3}}{\wprtp}\right) \nonumber\\
    &\iff
    \left(\sigma_{w 3} \wprtp < \rho_{w r_t} \sigma_{r_t 3} \wptwo\right)
    \land
    \left(\sigma_{r_t 3} \wprtp < \rho_{w r_t} \sigma_{w 3} \rtptwo\right) \nonumber\\
\end{align}
so that the \gls{rhs} is greater and the bound is less restrictive.
If we assume $\lambda_\theta = \lambda_r$ and $\lambda_{w\theta} = \lambda_{wr}$,
then the model gains 5 new \gls{pdf} parameters:
$\delta$, $\lambda_w$, $\lambda_\theta$, $\lambda_{w \theta}$, $\lambda_{\theta r}$.
If we want the \gls{pdf} to revert to a single normal distribution in the limit of zero skewness,
then we need
\begin{align}
    \delta, \lambda_w, \lambda_r, \lambda_\theta,
    \lambda_{w r}, \lambda_{w \theta}, \lambda_{\theta r}
    \to 1,
\end{align}
as $Sk_w \to 0$.
That being said, we can have a look at \cref{eq:sk_hat_w_nondim} and take the limit.
\begin{align}
    \lim_{\delta \to 1}\left(\widehat{Sk}_w\right)
    =&\lim_{\delta \to 1}
    \left(
    \frac{1}{\tswfact^{3/2}}
    \frac{\wpthree}{\left(\wptwo \right)^{3/2}}
    \frac{1}{\left(\frac{1-\delta \lambda_w}{1-\delta}\right)^{3/2}}
    \frac{1}{1-\delta}
    \right)
    \nonumber \\
    =&\lim_{\delta \to 1}
    \left(
    \frac{1}{\tswfact^{3/2}}
    \frac{\wpthree}{\left(\wptwo \right)^{3/2}}
    \frac{(1-\delta)^{3/2}}{(1-\delta \lambda_w)^{3/2}}
    \frac{1}{1-\delta}
    \right)
    \nonumber \\
    =&\lim_{\delta \to 1}
    \left(
    \frac{1}{\tswfact^{3/2}}
    \frac{\wpthree}{\left(\wptwo \right)^{3/2}}
    \frac{\cancelto{0}{(1-\delta)^{1/2}}}{(1-\delta \lambda_w)^{3/2}}
    \right)
    \nonumber \\
    =& \quad 0
\end{align}
Hence, if $\delta \to 1$ we get $Sk_w \to 0$.
Of course, one needs to pay attention when computing those equations in the code,
since we could run into a division by zero error,
depending on which values are computed first.

Also, we want to see how the \enquote{main} equations behave in the limit of $\delta \to 1$
as well as any $\lambda \to 1$.
For those limits, we assume all lower-order moments as constants because they are just given to us in the code.


\section{Limits for \texorpdfstring{$\delta \to 1$}{delta to 1}}
\label{sec:limits-for-delta-to-1}

We start with letting $\delta \to 1$.
Also, we are evaluating the following proposed fit for $\lambda_w$:
\begin{align}
    \lambda_w = \frac{1-\sqrt[3]{1-\delta}}{\delta},
\end{align}
where we treat $\lambda_w$ as a constant until the final step.

\subsection{Limit for \texorpdfstring{$\wpfour$}{wprime4bar} as \texorpdfstring{$\delta$}{delta} goes to 1}
\label{subsec:limit-for-wprime4bar-as-delta-goes-to-1}

\begin{align}
    \label{eq:limit_wprime4_delta_to_1}
    \lim_{\delta \to 1}\left(\overline{w'^4}\right)
    = \left(\wptwo\right)^2 \lim_{\delta \to 1} \left(\frac{ (1-\delta \lambda_w)^2}{1-\delta}\right) +
    \frac{\left(\overline{w'^3}\right)^2}{\wptwo} \lim_{\delta \to 1}\left(\frac{1}{1-\delta \lambda_w} \right) +
    \lim_{\delta \to 1}\left(\delta 3 \lambda_w^2 \left(\wptwo\right)^2\right)
\end{align}
Looking at this limit (for the calculation refer to \cref{sec:calculation-for-the-limit-for-wprime4bar-as-delta-goes-to-1}),
where we held the lower-order moments fixed and not depending on $\delta$,
unfortunately, we see that the fourth-order moment of $w$ diverges.
We try the proposed fit for $\lambda_w$:
\begin{align}
    \label{eq:limit_wprime4_delta_to_1_lambda_w_fit}
    \lim_{\delta \to 1}\left(\overline{w'^4}\right)
    &= \left(\wptwo\right)^2
    \lim_{\delta \to 1} \left(\frac{(1 - \left(1-\cancelto{0}{\sqrt[3]{1-\delta}}\quad\right)^2}{1-\delta}\right)
    \\
    &+ \frac{\left(\overline{w'^3}\right)^2}{\wptwo}
    \lim_{\delta \to 1}\left(\frac{1}{1-\left(1 - \left(1-\cancelto{0}{\sqrt[3]{1-\delta}}\quad\right) \right)} \right)
    \\
    &+ 3
    \lim_{\delta \to 1}\left(\left( 1 - \left(1-\cancelto{0}{\sqrt[3]{1-\delta}}\quad\right) \right)^2 \left(\wptwo\right)^2\right)
    \\
    &= \frac{\left(\overline{w'^3}\right)^2}{\wptwo}
\end{align}
At least, we have a convergence to a ratio between the squared skewness and the variance now.

\subsection{Limit for \texorpdfstring{$\wptwothlp$}{wprime2thetalbar} as \texorpdfstring{$\delta$}{delta} goes to 1}
\label{subsec:limit-for-wprime2thetalbar-as-delta-goes-to-1}

\begin{align}
    \label{eq:limit_wprime2thetalbar_delta_to_1}
    \lim_{\delta \to 1}\left(\wptwothlp\right)
    = \frac{\overline{w'^3} \cdot \overline{w'\theta_l'}}{\overline{w'^2}}
    \lim_{\delta \to 1}
    \left(\frac{1 - \delta \lambda_{w\theta}}{1-\delta \lambda_w}\right)
\end{align}
This limit (for the calculation refer to \cref{sec:calculation-for-the-limit-for-wprime2thetalprimebar-as-delta-goes-to-1})
is suggesting that if we want to have a limit of 0 as $\delta \to 1$ for $\wptwortp$ we should set $\lambda_{w\theta} = 1$.
Otherwise, if we set $\lambda_w = 0$, we get a limit that depends on the ratio between $\wpthree \cdot \wpthlp$ and $\wptwo$.

\subsection{Limit for \texorpdfstring{$\wptwothlp$}{wprimethetaltwobar} as \texorpdfstring{$\delta$}{delta} goes to 1}
\label{subsec:limit-for-wprimethetal2bar-as-delta-goes-to-1}

\begin{align}
    \label{eq:limit_wprimethetalprime2bar_delta_to_1}
    \lim_{\delta \to 1}\left(\wptwothlp\right)
    = \frac{1}{3}\left(
    \frac{2 \wpthree \left(\wpthlp \right)^2}{\left(\wptwo \right)^2} \lim_{\delta \to 1}
    \left(\left(\frac{1-\delta \lambda_{w \theta}}{1-\delta \lambda_w}\right)^2\right) +
    \frac{\wptwo \; \thlptwo}{\wpthlp}
    \lim_{\delta \to 1}
    \left(\frac{1-\delta \lambda_w}{1-\delta \lambda_{w \theta}}\right)
    \right)
\end{align}
This limit (for the calculation refer to \cref{sec:calculation-for-the-limit-for-wprimethetalprime2bar-as-delta-goes-to-1})

\subsection{Limit for \texorpdfstring{$\wprtpthlp$}{wprimertprimethetalbar} as \texorpdfstring{$\delta$}{delta} goes to 1}
\label{subsec:limit-for-wprimertprimethetalbar-as-delta-goes-to-1}

\begin{align}
    \label{eq:limit_wprimertprimethetalbar_delta_to_1}
    \lim_{\delta \to 1}\left(\wprtpthlp\right)
    &= \lim_{\delta \to 1}
    \left(
    (1 - \delta) \alpha (w_1 - \overline{w}) \left[
        \left(r_{t1} - \overline{r_t}\right)
        \left(\theta_{l1} - \overline{\theta_l}\right) +
        r_{r_t \theta_l} \sigma_{r_{t1}} \sigma_{\theta_{l1}}
        \right] \right.\nonumber\\
    &\quad\quad\quad+ \left.(1 - \delta) (1 - \alpha) (w_2 - \overline{w}) \left[
        \left(r_{t2} - \overline{r_t}\right)
        \left(\theta_{l2} - \overline{\theta_l}\right) +
        r_{r_t \theta_l} \sigma_{r_{t2}} \sigma_{\theta_{l2}}
        \right]
    \right) \nonumber\\
    &=\quad 0
\end{align}