\chapter{Asymptotics}\label{ch:asymptotics}

Once we defined all functions,
we see that we want certain behaviors for certain values as well as
there is a need to restrict some parameter values.

We start with the \enquote{obvious} restrictions for the \gls{pdf} parameters.
The mixture fractions $\alpha$ and $\delta$ are meant to be $\alpha \in [0,1]$ and $\delta \in [0,1)$.
But since the code tries to simplify a lot of things,
the binormal representation also does not revert back to a single normal distribution.
Therefore, we have $\alpha \in (0,1)$ due to the code.
The restriction to $\delta$ makes sense in a way
that we do not really want just the third normal to predict the whole shape.
Also, most of the formulas, e.g. \cref{eq:sk_hat_w_nondim} have a $1-\delta$ in the denominator.

In \cref{sec:derivation-of-trinormal-closures-by-transformation-of-binormal-closures},
we saw,
how the transformation between the sum of two normal distributions
and the sum of three normal distributions are working.
From those transformations, we see that we want
\begin{table}[!htb]
    \centering
    \begin{tabular}{llll}
        &
        $0 < \delta\lambda_w < 1$, &
        $0 < \delta\lambda_\theta < 1$, &
        $0 < \delta\lambda_r < 1$, \\
        $\iff$ &
        $0 < \delta\frac{\sigma_{w3}^2}{\wptwo} < 1$, &
        $0 < \delta\frac{\sigma_{\theta_l 3}^2}{\thlptwo} < 1$, &
        $0 < \delta\frac{\sigma_{r_t 3}^2}{\rtptwo} < 1$, \\
        $\iff$ &
        $0 < \delta\sigma_{w3}^2 < \wptwo$, &
        $0 < \delta\sigma_{\theta_l 3}^2 < \thlptwo$, &
        $0 < \delta\sigma_{r_t 3}^2 < \rtptwo$.
    \end{tabular}
    \label{tab:table_asymp_2}
\end{table}

That is, for instance,
that the variance of $w$ over the whole \gls{pdf} has to be strictly greater than $\delta$ times
the squared standard deviation in $w$ of the third normal distribution.

For realizability, it turns out that we want
$-1 < \widehat{c}_{w \theta_l}, \widehat{c}_{w r_t}, \widehat{c}_{r_t \theta_l} < 1$.
This is for instance:
\begin{align}
    c_{w r_t}^2 &< \tswfact \left(
    \frac{(1-\delta \lambda_w)(1-\delta \lambda_r)}{(1-\delta \lambda_{w r})^2}
    \right)
\end{align}
So it might be safer to set
\begin{align}
    \label{eq:lambda_corr_ineq}
    \lambda_w , \lambda_r < \lambda_{w r}
    \iff
    \frac{\sigma_{w 3}^2}{\wptwo}, \frac{\sigma_{r_t 3}^2}{\rtptwo} <
    \frac{\rho_{w r_t} \sigma_{w 3} \sigma_{r_t 3}}{\wprtp}
\end{align}
so that the \gls{rhs} is greater and the bound is less restrictive.
If we assume $\lambda_\theta = \lambda_r$ and $\lambda_{w\theta} = \lambda_{wr}$,
then we have 5 new \gls{pdf} parameters:
$\delta$, $\lambda_w$, $\lambda_\theta$, $\lambda_{w \theta}$, $\lambda_{\theta r}$.
If we want the \gls{pdf} to revert to a single normal distribution in the limit of zero skewness,
then we need
\begin{align}
    \delta, \lambda_w, \lambda_r, \lambda_\theta,
    \lambda_{w r}, \lambda_{w \theta}, \lambda_{\theta r}
    \to 1,
\end{align}
as $Sk_w \to 0$.
That being said, we can have a look at \cref{eq:sk_hat_w_nondim} and take the limit.

\begin{align}
    &\lim_{\delta \to 1, \lambda_w \to 1}
    \frac{1}{\tswfact^{3/2}}
    \frac{\wpthree}{\left(\wptwo \right)^{3/2}}
    \frac{1}{\left(\frac{1-\delta \lambda_w}{1-\delta}\right)^{3/2}}
    \frac{1}{1-\delta} \nonumber \\
    =&\lim_{\delta \to 1, \lambda_w \to 1}
    \frac{1}{\left(1-\frac{\sigma_w (1-\delta)^{1/2}}
    {\left(\wptwo\right)^{1/2} \left(1-\delta\lambda_w\right)^{1/2}}\right)^{3/2}}
    \frac{\wpthree}{\left(\wptwo \right)^{3/2}}
    \frac{(1-\delta)^{3/2}}{(1-\delta \lambda_w)^{3/2}}
    \frac{1}{1-\delta} \nonumber \\
    =&\lim_{\delta \to 1, \lambda_w \to 1}
    \frac{1}{
        \left(\frac{\left(\wptwo\right)^{1/2} \left(1-\delta\lambda_w\right)^{1/2} - \sigma_w (1-\delta)^{1/2}}
        {\left(\wptwo\right)^{1/2} \left(1-\delta\lambda_w\right)^{1/2}}\right)^{3/2}
    }
    \frac{\wpthree}{\left(\wptwo \right)^{3/2}}
    \frac{(1-\delta)^{1/2}}{(1-\delta \lambda_w)^{3/2}}
    \nonumber \\
    =&\lim_{\delta \to 1, \lambda_w \to 1}
    \frac{\left(\left(\wptwo\right)^{1/2} \left(1-\delta\lambda_w\right)^{1/2}\right)^{3/2}}
    {\left(\wptwo\right)^{1/2} \left(1-\delta\lambda_w\right)^{1/2} - \sigma_w (1-\delta)^{1/2}}
    \frac{\wpthree}{\left(\wptwo \right)^{3/2}}
    \frac{(1-\delta)^{1/2}}{(1-\delta \lambda_w)^{3/2}}
    \nonumber \\
    =&\lim_{\delta \to 1, \lambda_w \to 1}
    \frac{\left(\wptwo\right)^{3/4} \left(1-\delta\lambda_w\right)^{3/4}}
    {\left(\wptwo\right)^{1/2} \left(1-\delta\lambda_w\right)^{1/2} - \sigma_w (1-\delta)^{1/2}}
    \frac{\wpthree}{\left(\wptwo \right)^{3/2}}
    \frac{(1-\delta)^{1/2}}{(1-\delta \lambda_w)^{3/2}}
    \nonumber \\
    =&\lim_{\delta \to 1, \lambda_w \to 1}
    \frac{
        \left(\wptwo\right)^{3/4} \left(1-\delta\lambda_w\right)^{3/4} \wpthree (1-\delta)^{1/2}
    }
    {
        \left(\wptwo\right)^2 (1-\delta \lambda_w)^2 -
        (1-\delta)^{1/2} (1-\delta \lambda_w)^{3/2} \sigma_w
    }
    \nonumber \\
    =&\lim_{\delta \to 1, \lambda_w \to 1}
    \frac{
        \left(\wptwo\right)^{3/4} \left(1-\delta\lambda_w\right)^{3/4} \wpthree (1-\delta)^{1/2}
    }
    {
        (1-\delta \lambda_w)
        \left(
        \left(\wptwo\right)^2 -
        \frac{(1-\delta)^{1/2} \sigma_w}{(1-\delta \lambda_w)^{1/2}}
        \right)
    }
    \nonumber \\
    =&\lim_{\delta \to 1, \lambda_w \to 1}
    \frac{
        \left(\wptwo\right)^{3/4} \wpthree (1-\delta)^{1/2}
    }
    {
        (1-\delta \lambda_w)^{1/4}
        \left(
        \left(\wptwo\right)^2 -
        \frac{(1-\delta)^{1/2} \sigma_w}{(1-\delta \lambda_w)^{1/2}}
        \right)
    }
    \nonumber\\
    =&\lim_{\delta \to 1, \lambda_w \to 1}
    \frac{
        \left(\wptwo\right)^{3/4} \wpthree (1-\delta)^{1/2} (1-\delta \lambda_w)^{-1/4}
    }
    {
        \frac{\left(\wptwo\right)^2 (1-\delta \lambda_w)^{1/2}
            - (1-\delta)^{1/2} \sigma_w}{(1-\delta \lambda_w)^{1/2}}
    }
    \nonumber\\
    =&\lim_{\delta \to 1, \lambda_w \to 1}
    \frac{
        \left(\wptwo\right)^{3/4} \wpthree (1-\delta)^{1/2} (1-\delta \lambda_w)^{1/4}
    }
    {
        \left(\wptwo\right)^2 (1-\delta \lambda_w)^{1/2}
        - (1-\delta)^{1/2} \sigma_w
    }
\end{align}

Hence, if $\delta \to 1$ we get $Sk_w \to 0$.
Of course, one needs to pay attention when computing those equations in the code,
since we could run into a division by zero error,
depending on which values are computed first.

Also, we want to see how the \enquote{main} equations behave in the limit of $\delta \to 1$ as well as any $\lambda \to 1$.
For those limits, we assume all lower-order moments as constants, because in the code, they are just given to us.
We start with letting $\delta \to 1$.

\section{Limit for \texorpdfstring{$\wpfour$}{wprime4bar} as \texorpdfstring{$\delta$}{delta} goes to 1}
\label{sec:limit-for-wprime4bar-as-delta-goes-to-1}

\begin{align}
    \label{eq:limit_wprime4_delta_to_1}
    \lim_{\delta \to 1}\left(\overline{w'^4}\right)
    = \left(\overline{w'^2}\right)^2
    \lim_{\delta \to 1}
    \left(\frac{
        \left(1-\delta\lambda_w\right)^2
    }{(1-\delta)}
    \right)
\end{align}

\section{Limit for \texorpdfstring{$\wptwothlp$}{wprime2thetalbar} as \texorpdfstring{$\delta$}{delta} goes to 1}
\label{sec:limit-for-wprime2thetalbar-as-delta-goes-to-1}

\begin{align}
    \label{eq:limit_wprime2thetalbar_delta_to_1}
    \lim_{\delta \to 1}\left(\wptwothlp\right)
\end{align}