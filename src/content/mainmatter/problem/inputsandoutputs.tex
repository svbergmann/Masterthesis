\section{Inputs and outputs of the verification procedure}
\label{sec:inputs-and-outputs-of-the-verification-procedure}

While defining inputs and outputs can seem challenging at first glance,
it is a crucial step towards understanding a system.

\subsection{Inputs and outputs of a forward run}
\label{subsec:inputs-and-outputs-of-a-forward-run}

When forecasting the weather (\enquote{forward run}),
the code provides us with a set of moment terms:
$\overline{w}$, $\overline{w'^2}$, $\overline{w'^3}$, $\overline{\theta_l}$, $\overline{w'\theta_l'}$,
$\overline{r_t}$, $\overline{w' r_t'}$, $\overline{\theta_l'^2}$, $\overline{r_t'^2}$, $\overline{r_t'\theta_l'}$.
These are the inputs.
From these inputs,
we want to determine certain parameters which describe the shape of the underlying \gls{pdf}.
Those \gls{pdf} parameters are standardized and some also normalized.
So we try to solve these \gls{pdf} parameters (13),
namely $\alpha$, $\widehat{w}_1$, $\widehat{w}_2$, $\tilde{\theta}_{l1}$, $\tilde{\theta}_{l2}$, $\tilde{r}_{t1}$,
$\tilde{r}_{t2}$, $\tilde{\sigma}_w$, $\tilde{\sigma}_{\theta_{l1}}$, $\tilde{\sigma}_{\theta_{l2}}$,
$\tilde{\sigma}_{r_{t1}}$, $\tilde{\sigma}_{r_{t2}}$, and $r_{r_t \theta_l}$.
All the formulas are listed in \cref{ch:formulas}.
Ultimately, the code needs to express even higher order moments such as $\overline{w'^2 \theta_l'}$
in terms of the lower order moments.
These higher order moments are the outputs in the \enquote{forward run}.

\subsection{Inputs and outputs of a backward run (verification direction)}
\label{subsec:inputs-and-outputs-of-a-backward-run-(verification-direction)}

Although a \enquote{forward run} models the higher order moments in terms of the lower order moments,
we want to verify these formulas,
namely \cref{eq:wp4}, \cref{eq:wp2thlp_solved}, and \cref{eq:wpthlp2_solved}.
To achieve this, we will take a more traditional approach, working in the \enquote{backward} direction.
This means we will:
\begin{enumerate}
    \item \emph{Specify the \gls{pdf} parameters:}
    Start by explicitly defining the parameters that characterize the underlying \gls{pdf}.
    \item \emph{Calculate the moments:}
    Once the \gls{pdf} is defined, we can then calculate the desired moments,
    such as $\overline{w}$, through integration.
\end{enumerate}
This can be done, e.g.\ by calculating the integral:
\begin{align}
    \overline{w}
    &= \int_{\mathbb{R}} \int_{\mathbb{R}} \int_{\mathbb{R}} w \cdot P_{tmg} \; dw dr_t d\theta_l,
\end{align}
where $P_{tmg}$ (\textbf{T}rivariate \textbf{M}ixture of \textbf{G}aussians)
is the \gls{pdf} of the sum of all three normal distributions.
Since some integrals are challenging to verify symbolically with SymPy,
we are using the quadrature method of SymPy to calculate the integrals
and choose arbitrary values for the inputs.
All of this can be seen in \cref{sec:numintsympy}.
