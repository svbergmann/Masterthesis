\section{Inputs and Outputs}\label{sec:inputsandoutputs}

While defining inputs and outputs can seem challenging at first glance, it's a crucial step towards understanding a system.
In this context, the code provides us with a set of moment terms: $\overline{w}$, $\overline{w'^2}$, $\overline{w'^3}$, $\overline{\theta_l}$, $\overline{w'\theta_l'}$, $\overline{r_t}$, $\overline{w' r_t'}$, $\overline{\theta_l'^2}$, $\overline{r_t'^2}$, $\overline{r_t'\theta_l'}$.
From these, we want to determine certain parameters which are describing the shape of the underlying \gls{pdf}.
Those parameters are standardized and some also normalized.
So we try to solve for 13 unknowns, namely $\alpha$, $\widehat{w}_1$, $\widehat{w}_2$, $\tilde{\theta}_{l1}$, $\tilde{\theta}_{l2}$, $\tilde{r}_{t1}$, $\tilde{r}_{t2}$, $\tilde{\sigma}_w$, $\tilde{\sigma}_{\theta_{l1}}$, $\tilde{\sigma}_{\theta_{l2}}$, $\tilde{\sigma}_{r_{t1}}$, $\tilde{\sigma}_{r_{t2}}$, and $r_{r_t \theta_l}$.
All the formulas are listed in \cref{ch:formulas}.

While the code seems to derive the desired moments from a set of parameters, we want to establish a formal mathematical foundation for these relationships.
To achieve this, we will take a more traditional approach, working in the \enquote{forward} direction.
This means we will:
\begin{enumerate}
    \item \emph{Define the \gls{pdf} parameters:}
    Start by explicitly defining the parameters that characterize the underlying \gls{pdf}.
    \item \emph{Calculate the moments:}
    Once the \gls{pdf} is defined, we can then calculate the desired moments, such as $\overline{w}$, through integration.
\end{enumerate}
This can be done, e.g.\ by calculating the integral:
\begin{align}
    \overline{w}
    &= \int_{\mathbb{R}} \int_{\mathbb{R}} \int_{\mathbb{R}} w \cdot P_{tmg} \; dw dr_t d\theta_l,
\end{align}
where $P_{tmg}$\footnote{\textbf{T}rivariate \textbf{M}ixture of \textbf{G}aussians} is the \gls{pdf} of the sum of all three Normal distributions.
Since some integrals are challenging to solve analytically, even with the help of SymPy, we are using the quadrature method to calculate the integrals and choose arbitrary values for the inputs.
All of this can be seen in \cref{sec:numintsympy}.
