\section{Closing PDEs}\label{sec:closing_pdes}

To get a better sense of why these integral checks are important for the model, it follows a brief description about what the code is actually doing in comparison to what we are trying to prove.
Firstly, we are dealing with a set of \glspl{pde} predicted by the \gls{CLUBB} model.
These equations require closure, which essentially means finding a way to express them solely in terms of known quantities.
For an example of such a \gls{pde}, you can refer to the following\cite[p. 21]{larson2022clubbsilhs}:
\begin{align*}
    \frac{\partial \overline{w'r_t'}}{\partial t}
    &= -\frac{\partial \overline{w'r_t'}}{\partial z}
    - \frac{1}{\rho_s} \frac{\partial \rho_s \overline{w'^2r_t'}}{\partial z}
    - \overline{w'^2} \frac{\partial \overline{r_t'}}{\partial z}
    - \overline{w'r_t'} \frac{\partial \overline{w}}{\partial z}
    + \ldots
\end{align*}
To close equations like this, \gls{CLUBB} needs some initial conditions, as well as boundary conditions.
This is supposed to be just a site note because it is not important for the purpose of this thesis.
Once we have those conditions, the moments on the \gls{rhs} need to be calculated as fast and efficient as possible.
That is because the model steps forward in time and needs to calculate those moments for each unclosed, prognosted equation.
Therefore, \gls{CLUBB} prefers mathematically simpler models over greater variability to achieve exactly this.