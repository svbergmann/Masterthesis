\section{Closing turbulence pdes by integration over a pdf}
\label{sec:closing-turbulence-pdes-by-integration-over-a-pdf}

The \gls{CLUBB} model relies on a set of \glspl{pde} to represent atmospheric processes.
These equations require closure,
implying the expression of all terms solely in terms of known quantities.
This closure process often involves integrals,
and verifying their analytical solutions ensures the model's mathematical integrity.
For instance, consider the following prognostic \gls{pde}~\autocite[p. 21]{larson2022clubbsilhs}:
\begin{align*}
    \frac{\partial \overline{w'\theta_l'}}{\partial t}
    &= -\overline{w}\frac{\partial \overline{w'\theta_l'}}{\partial z}
    - \frac{1}{\rho_s} \frac{\partial \rho_s \overline{w'^2 \theta_l'}}{\partial z}
    - \overline{w'^2} \frac{\partial \overline{\theta_l'}}{\partial z}
    - \overline{w'\theta_l'} \frac{\partial \overline{w}}{\partial z}
    + \ldots
\end{align*}
While the details of initial and boundary conditions are essential
for formally closing the prognostic equations in \gls{CLUBB} (omitted for brevity),
this section emphasizes the importance of efficiently calculating moments
on the \gls{rhs} of these equations.
That is because the model steps forward in time
and therefore needs the repeated calculation of moments
for each unclosed prognostic equation at every time step.
Those already expensive computational steps need to use mathematically simpler moment representations
of e.g. $\overline{w'^2 \theta_l'}$,
even if they introduce slight limitations
in capturing the full variability of the underlying atmospheric state.