\section{Steps for checking the formulas}\label{sec:steps_for_checking}

This section outlines a general approach for verifying all of those integral expressions
employed within the \gls{CLUBB} model.
This approach ensures the accuracy of the computed moment relationships.
We try to reference to the following steps in \cref{ch:intsympy} where there are actual examples stated.
One way, or at least the way, which is used in this document is the following,
where we always want to check if \gls{lhs} equals \gls{rhs}:

\begin{enumerate}
    \item\label{itm:checkingstep_1}
    Choose \emph{dimensional} parameters (parameters without any tilde or hat) that determine the \gls{pdf},
    i.e.\ choose dimensional \gls{pdf} parameters, e.g. $\sigma_{w 3}$.
    Then the \gls{pdf} is known and any moments of it can be calculated by integration.

    \item\label{itm:checkingstep_2}
    Calculate the means, e.g. $\overline{w} = \mathbb{E}[w]$ by integration over the \gls{pdf}.
    The formula for $\overline{w}$ (\cref{eq:w_bar}) in terms of the \gls{pdf} parameters can be checked.

    \item\label{itm:checkingstep_3}
    Once the means are known, we calculate the central variances,
    e.g.  $\overline{w'^2} = \overline{(w-\overline{w})^2}$ by integration over the \gls{pdf}.
    The formula for $\overline{w'^2}$ (\cref{eq:wp2_bar}) in terms of the \gls{pdf} parameters can be checked.

    \item\label{itm:checkingstep_4}
    Once the variances, e.g. $\overline{w'^2}$, are known,
    then \emph{non-dimensional} \gls{pdf} parameters such as $\lambda_w$ can be calculated by their definitions.

    \item\label{itm:checkingstep_5}
    We can also calculate the covariances by 2D integration over a 2D \gls{pdf}.
    Again, our formulas in terms of \gls{pdf} parameters can be checked.

    \item\label{itm:checkingstep_6}
    Finally, we can calculate the higher order moments,
    i.e. $\overline{w'^4}$ (\cref{eq:wp4}) or $\overline{w'^2 \theta_l'}$ (\cref{eq:wp2thlp_solved})
    or $\overline{w' \theta_l'^2}$ (\cref{eq:wpthlp2_solved}),
    by integration over the \gls{pdf}.
    Again, we can check if \gls{lhs} equals \gls{rhs}.

%    \item\label{itm:checkingstep_7}
%    The integral $\overline{w'\theta_l'^2}$ is complicated.
%    If it is written in terms of $\overline{\theta_l'^3}$ (\cref{eq:wpthlp2_solved}),
%    then $\overline{\theta_l'^3}$ can be integrated (\cref{eq:theta_l_3_bar})
%    and substituted into the \gls{rhs} of \cref{eq:wpthlp2_solved}.
%    If $\wpthlptwo$ is written in terms of $\beta$ (\cref{eq:wpthlp2_beta}),
%    then we need to integrate to find $\thlpthree$,
%    back out $\beta$ from \cref{eq:thlp3_beta},
%    and finally substitute $\beta$ into \cref{eq:wpthlp2_solved}.
%
%    \item\label{itm:checkingstep_8}
%    To verify \cref{eq:wpqtpthlp_beta} for $\wprtpthlp$,
%    we use \cref{eq:w_prime_r_t_prime_t_l_prime_bar}.
\end{enumerate}
