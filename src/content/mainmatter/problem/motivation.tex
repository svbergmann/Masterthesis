\section{Motivation}\label{sec:motivation}

As being said in the \cref{ch:introduction},
we try to describe more possible shapes with adding a third \gls{pdf}.
To illustrate that, we plotted some of the shapes which are now possible but were not possible before.
To be able to draw those plots, we are just using two variables, $w$, the upward wind,
and $\theta_l$, the liquid water potential temperature.
To illustrate how the binormal model handles strong winds,
let us consider a scenario with a strong updraft at $w_1$, as well as a strong downdraft at $w_2$.
The way the current binormal model would model this could look like \cref{fig:plot1}.

\begin{figure}[!htb]
    \centering
    \includegraphics[width=.5\textwidth]{include/figures/plot1}
    \caption{Binormal plot for two strong upward winds}
    \label{fig:plot1}
    With fixed variables: $w_1 = 5$, $w_2 = -5$, $\theta_{l1} = 5$, $\theta_{l2} = -5$,
    $\alpha = 0.5$, $\sigma_w = 2$, $\sigma_{\theta_{l1}} = 2$, $\sigma_{\theta_{l1}} = 2$.
\end{figure}

However, this bimodal distribution (\cref{fig:plot1}) doesn't accurately reflect reality.
In nature, we would not expect such a sharp jump between the strong upward winds at $w_1$ and $w_2$.
There would likely be some weaker upward winds present in between.
The current binormal model can attempt to capture this smoother transition
by simply increasing the standard deviations of both wind
and liquid water potential temperature distributions.
This results in a broader distribution with a connection between the two peaks,
as shown in \cref{fig:plot2}.

\begin{figure}[!htb]
    \centering
    \includegraphics[width=.5\textwidth]{include/figures/plot2}
    \caption{Binormal plot for two strong upward winds with increased standard deviations}
    \label{fig:plot2}
    With fixed variables: $w_1 = 5$, $w_2 = -5$, $\theta_{l1} = 5$, $\theta_{l2} = -5$,
    $\alpha = 0.5$, $\sigma_w = 5$, $\sigma_{\theta_{l1}} = 5$, $\sigma_{\theta_{l1}} = 5$.
\end{figure}

Seeing \cref{fig:plot2},
the issue with having some values in the middle is slightly fixed
but the general width of the normals was increased, too.
Since \gls{CLUBB} also has the simplification that there is no correlation between $w$ and $\theta_l$,
and $w$ and $r_t$, obviously one cannot just increase it.
Therefore, the idea is to add this third normal,
which actually has correlation between all three variables
and especially in the bivariate case, between $w$ and $\theta_l$.
\Cref{fig:plot2} would then change to \cref{fig:plot3}.

\begin{figure}[!htb]
    \centering
    \begin{tabular}{cc}
        \multicolumn{1}{c}{\includegraphics[width=0.48\textwidth]{include/figures/plot3_1}} &
        \multicolumn{1}{c}{\includegraphics[width=0.48\textwidth]{include/figures/plot3_2}} \\
        \multicolumn{1}{c}{\includegraphics[width=0.48\textwidth]{include/figures/plot3_3}} &
        \multicolumn{1}{c}{\includegraphics[width=0.48\textwidth]{include/figures/plot3_4}} \\
    \end{tabular}
    \caption{Trinormal plot for two strong upward winds with varying $\delta$}
    \label{fig:plot3}
    With fixed variables: $w_1 = 5$, $w_2 = -5$, $\theta_{l1} = 5$, $\theta_{l2} = -5$,
    $\alpha = 0.5$, $\sigma_w = 2$, $\sigma_{\theta_{l1}} = 2$,  $\sigma_{\theta_{l2}} = 2$,
    $\sigma_{w3} = 2$, $\sigma_{3\theta_l} = 2$, $\rho_{w\theta_l} = 0.5$.
\end{figure}

Now, one can easily model something like the described shape,
as illustrated in \cref{fig:plot3}.
Also, some other (maybe weird) shapes are now possible,
just like the one in \cref{fig:plot4}.

\begin{figure}[!htb]
    \centering
    \includegraphics[width=.5\textwidth]{include/figures/plot4}
    \caption{Trinormal plot for two strong upward winds with a third peak in the middle}
    \label{fig:plot4}
    With fixed variables: $w_1 = 5$, $w_2 = -5$, $\theta_{l1} = 5$, $\theta_{l2} = -5$,
    $\alpha = 0.5$, $\delta=0.5$, $\sigma_w = 2$, $\sigma_{\theta_{l1}} = 2$,
    $\sigma_{\theta_{l2}} = 2$, $\sigma_{w3} = 2$, $\sigma_{\theta_l 3} = 2$,
    $\rho_{w\theta_l} = 0.5$.
\end{figure}