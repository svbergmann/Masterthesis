\section{Expressions for higher-order moments in terms of pdf parameters}
\label{sec:expressions-for-higher-order-moments-in-terms-of-pdf-parameters}

Upon determining the \gls{pdf} parameters,
we gain the ability to compute all higher-order moments associated with the distribution.
These moments play a crucial role for closing the already described \glspl{pde}.
The symbolic calculation of higher-order moments can be achieved
through integration over the specified \gls{pdf}.
Formulas for calculating various higher-order moments within the context of a binormal \gls{pdf}
are readily available in the literature~\cite{larson2005using}.

We state the transformed formulas needed for closure in the following.

\begin{align}
    \label{eq:wp4_div_wp2_2}
    \frac{1}{\tswfact^2} \frac{(1-\delta)}{(1-\delta \lambda_w)^2} \frac{\wpfour}{\left(\wptwo \right)^2}
    &= \alpha \left[\widehat{w}_1^4 +
    6 \widehat{w}_1^2 \frac{\tsw^2}{\tswfact} +
    3 \frac{\tsw^4}{\tswfact^2} \right] \nonumber \\
    &+ (1 - \alpha) \left[\widehat{w}_2^4 +
    6 \widehat{w}_2^2 \frac{\tsw^2}{\tswfact} +
    3 \frac{\tsw^4}{\tswfact^2} \right] \nonumber \\
    &+ \frac{1}{\tswfact^2} \frac{(1-\delta)}{(1-\delta \lambda_w)^2} \delta 3 \lambda_w^2,
\end{align}
\begin{align}
    \label{eq:wp2thlp_div_wp2thl2}
    \frac{1}{\tswfact} \frac{(1-\delta)^{1/2}}{(1-\delta \lambda_w) (1-\delta \lambda_\theta)^{1/2}}
    \frac{\wptwothlp}{\wptwo \left(\thlptwo \right)^{1/2}}
    &= \alpha \left[\widehat{w}_1^2 + \frac{\tsw^2}{\tswfact} \right] \tilde{\theta}_{l1} \nonumber\\
    &+ (1-\alpha) \left[\widehat{w}_2^2 + \frac{\tsw^2}{\tswfact}\right] \tilde{\theta}_{l2},
\end{align}
\begin{align}
    \label{eq:wpthlp2_div_wp2_thlp2}
    \frac{1}{\tswfact^{1/2}} \frac{(1-\delta)^{1/2}}{(1-\delta \lambda_w)^{1/2} (1-\delta \lambda_\theta)}
    \frac{\wpthlptwo}{\left(\wptwo\right)^{1/2}\thlptwo}
    = \alpha  \widehat{w}_1  \left( \tilde{\theta}_{l1}^2 + \tilde{\sigma}_{\theta_{l1}}^2 \right)
    + (1-\alpha) \widehat{w}_2 \left( \tilde{\theta}_{l2}^2 + \tilde{\sigma}_{\theta_{l2}}^2 \right),
\end{align}
\begin{align}
    \label{eq:wprtpthlp_div_wp2_thp2_rt2}
    \frac{1}{\tswfact^{1/2}}
    \frac{(1-\delta)^{1/2}}
    {(1-\delta \lambda_w)^{1/2} (1-\delta \lambda_\theta)^{1/2}(1-\delta \lambda_{q_t})^{1/2}}
    \frac{\wprtpthlp}
    {\left( \wptwo \right)^{1/2}\left(\rtptwo \right)^{1/2} \left( \thlptwo \right)^{1/2}} \nonumber \\
    = \alpha \widehat{w}_1 \left(\tilde{r}_{t1} \tilde{\theta}_{l1} +
    r_{r_t \theta_l} \tilde{\sigma}_{r_{t1}} \tilde{\sigma}_{\theta_{l1}} \right) +
    (1-\alpha) \widehat{w}_2 \left(\tilde{r}_{t2} \tilde{\theta}_{l2} +
    r_{r_t \theta_l} \tilde{\sigma}_{r_{t2}} \tilde{\sigma}_{\theta_{l2}} \right)
\end{align}
Equations for $\wptwortp$ and $\wprtptwo$ are similar to \cref{eq:wp2thlp_div_wp2thl2}
and \cref{eq:wpthlp2_div_wp2_thlp2} by replacing $\theta_l$ with $r_t$ everywhere.