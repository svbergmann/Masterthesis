\section{Addressing skewness and achieving closure}
\label{sec:addressing-skeweness-and-achieving-closure}

Closing the system of prognostic equations within \gls{CLUBB} needs the specification of
the skewness terms $Sk_{\theta_l}$ and $Sk_{r_t}$.
These skewness values appear in the solutions for $\tilde{\sigma}_{\theta_l 1}$ (\cref{eq:sigma_tilde_theta1_solved})
and $\tilde{\sigma}_{\theta_l 2}$ (\cref{eq:sigma_tilde_theta2_solved}), respectively.
Traditionally, these skewness terms could be treated as prognostic variables,
requiring their own prognostic equations and adding to the overall intense computation.

Therefore, the paper~\citetitle{larson2005using} by~\citeauthor{larson2005using} which this work is based on,
proposes an alternative approach that uses a diagnostic formula for skewness.
The formula provides a reasonable estimate of the skewness terms
based on the readily available prognostic moments,
avoiding the need for dedicated prognostic equations for skewness.
This strategy results in closure of the system of equations
while maintaining a computationally tractable model.
The proposed formula is the following:
\begin{align}
    \label{eq:Sk_hat_thl_beta}
    \widehat{Sk}_{\theta_l}
    &= \widehat{Sk}_w \widehat{c}_{w \theta_l} \left[\beta + (1-\beta) \widehat{c}_{w \theta_l}^2 \right],
\end{align}
which is similar for $\widehat{Sk}_{r_t}$ by again just replacing $r_t$ with $\theta_l$.
They define a parameter $\beta$ which is dimensionless.
We also solve for $\beta$ because we are going to need the equation later on to show that other equations are true.
\begin{align}
    \label{eq:beta}
    \implies \beta
    &=\frac{
        \frac{\widehat{Sk}_{\theta_l}}{\widehat{Sk}_w \widehat{c}_{w \theta_l}} - \widehat{c}_{w \theta_l}^2}
    {1 - \widehat{c}_{w \theta_l}^2}
\end{align}
\Cref{eq:Sk_hat_thl_beta} presents a diagnostic formula for estimating the skewness of $\theta_l$.
This formula offers a physically intuitive relationship.
It proposes a proportionality between $Sk_{\theta_l}$ and $Sk_w$.
However, it's crucial to acknowledge the limitations in this diagnostic.
The formula suggests that an increase in the parameter $\beta$ leads to a larger magnitude of $Sk_{\theta_l}$.
This translates to a \gls{pdf} with a more extended tail in the $\theta_l$ domain.
Furthermore, the formula captures the behavior when $w$
and $\theta_l$ are correlated.
That is, positive skewness in $w$ leads to positive skewness in $\theta_l$ (positive correlation),
and vice versa (negative correlation).
However, it is important to mention that real-world large eddy simulations
may show deviations from this simplified relationship.

Another limitation appears when either $Sk_w$
or the covariance between $w$ and $\theta_l$ ($c_{w \theta_l}$) approaches zero.
The formula predicts a vanishing $Sk_{\theta_l}$ in these scenarios,
which may not always be true.

Finally, the diagnostic approach allows for the magnitude of $\left| Sk_{\theta_l} \right|$
to be either smaller or larger than $\left| Sk_w \right|$.
This behavior depends on the interplay between the variance of $w$ ($\tsw^2$),
the covariance ($c_{w \theta_l}$),
and the parameter $\beta$.
This highlights the potential for discrepancies between the estimated skewness
and the actual skewness observed in real-world atmospheric data.

To summarize, \cref{eq:Sk_hat_thl_beta} offers a computationally efficient method for skewness estimation,
but it comes with limitations.
While it captures some key aspects of the relationship between the skewness in $w$
and the skewness in $\theta_l$,
one should be aware of deviations from its predictions.

We proceed with using \cref{eq:sk_hat_theta_l_nondim} for $\widehat{Sk}_{\theta_l}$
and find the following relationships~\cite{larson2005using}
for $\tilde{\sigma}_{\theta_l 1}^2$ and $\tilde{\sigma}_{\theta_l 2}^2$:
\begin{align}
    \label{eq:sigma_theta1_beta}
    \tilde{\sigma}_{\theta_l 1}^2
    &= \frac{\left(1 - \widehat{c}_{w \theta_l}^2\right)}{\alpha} \left[\frac{1}{3} \beta + \alpha \left(1 - \frac{2}{3} \beta\right)\right],
\end{align}
and
\begin{align}
    \label{eq:sigma_theta2_beta}
    \tilde{\sigma}_{\theta_l 2}^2
    &= \frac{\left(1 - \widehat{c}_{w \theta_l}^2\right)} {1 - \alpha} \left\{1 - \left[\frac{1}{3}\beta + \alpha \left(1 - \frac{2}{3} \beta \right)\right]\right\}.
\end{align}

By using the previously stated expressions for the standard deviations
(equations~\eqref{eq:sigma_theta1_beta} and~\eqref{eq:sigma_theta2_beta},
with their $r_t$ counterparts),
we can substitute them into the formula for the correlation between
$r_t$ and $\theta_l$ (\cref{eq:r_r_t_theta_l}).
This substitution leads to a more concise representation.
\begin{align}
    \label{eq:r_r_t_theta_l_beta}
    r_{r_t \theta_l}
    &= \frac{c_{r_t \theta_l} - \widehat{c}_{w r_t} \widehat{c}_{w \theta_l}}
    {\left(1 - \widehat{c}_{w r_t}^2\right)^{1/2} \left(1 - \widehat{c}_{w \theta_l}^2\right)^{1/2}},
\end{align}
where the correlation of $r_t$ and $\theta_l$ within the individual normal distributions is $r_{r_t \theta_l}$,
and $c_{r_t \theta_l}$ represents the total correlation across the entire trinormal \gls{pdf}.

\gls{CLUBB} chooses a specific formula for the $w$-\enquote{width} of the individual normals,
which is the following.
\begin{align}
    \label{eq:sigma_tilde_w_restricted}
    \tsw^2
    &= \gamma \left[1 - \max(c_{w \theta_l}^2, c_{w r_t}^2) \right].
\end{align}
The stated formula incorporates the covariances
-- $c_{w r_t}^2$ and $c_{w \theta_l}^2$ --
and ensures that $\tilde{\sigma}_w^2$ is bounded between 0 and 1.
That is because a new dimensionless parameter $\gamma$ was introduced
whose domain is $[0, 1)$.
This mathematical safeguard guarantees
that even when the original covariances ($c_{w r_t}^2$ and $c_{w \theta_l}^2$)
become large in magnitude,
the derived variance remains within a realistic and mathematically tractable bound
($0 \leq \widehat{c}_{w \theta_l}^2, \widehat{c}_{w r_t}^2$ < 1).
It also follows, that $\tilde{\sigma}_{r_{ti}}$, and $r_{r_t \theta_l}$ remain in a realistic bound.