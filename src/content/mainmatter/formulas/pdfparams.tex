\section{Finding pdf parameters in terms of moments}\label{sec:pdfparams}

We now select a particular member of the normal mixture family
by mapping the prognosed moments to the \gls{pdf} parameters.
In other words, we invert \cref{eq:w_bar_nondim} - \cref{eq:c_hat_r_t_theta_l_nondim}
in order to find the set of \gls{pdf} parameters
that guarantees that the resulting \gls{pdf} has moments that correspond to the prognosed ones.
The inversion is non-trivial because the equations are non-linear in the \gls{pdf} parameters.
However, the \gls{pdf} (\cref{eq:normal_mix_pdf}) is simple enough to permit an analytic solution.

The solution procedure is as follows.
\begin{enumerate}
    \item Solve for the \gls{pdf} parameters $\alpha$, $\widehat{w}_1$,
    and $\widehat{w}_2$ from the moment equations for $\overline{w}$ (\cref{eq:w_bar_nondim}),
    $\wptwo$ (\cref{eq:wp2_bar_non_dim}), $\wpthree$ (\cref{eq:sk_hat_w_nondim}):
    \begin{align}
        \label{eq:alpha_solved}
        \alpha
        &= \frac{1}{2}\left[1 - \widehat{Sk}_w \sqrt{\frac{1}{4 + \widehat{Sk}_w^2}}\right],
    \end{align}
    \begin{align}
        \label{eq:w1_solved}
        \widehat{w}_1
        &= \sqrt{\frac{1-\alpha}{\alpha}},
    \end{align}
    \begin{align}
        \label{eq:w2_solved}
        \widehat{w}_2
        &= -\sqrt{\frac{\alpha}{1-\alpha}}.
    \end{align}
    Without loss of generality, it has been chosen to set $\widehat{w}_1 > \widehat{w}_2$.

    \item \Cref{eq:alpha_solved} implies that $\widehat{Sk}_w$ is determined solely by $\alpha$:
    \begin{align}
        \label{eq:sk_w_alpha}
        \widehat{Sk}_w
        &= \frac{1-2\alpha}{\sqrt{\alpha(1-\alpha)}}.
    \end{align}

    \item We can obtain $\Tilde{\theta}_{l1}$ and $\Tilde{\theta}_{l2}$ from \cref{eq:thlp_bar_nondim}
    for $\overline{\theta_l}$, and \cref{eq:c_hat_w_theta_l_nondim} for $\wpthlp$:
    \begin{align}
        \label{eq:thl1_tilde_solved}
        \tilde{\theta}_{l1}
        &= -\frac{\widehat{c}_{w \theta_l}}{\widehat{w}_2},
    \end{align}
    \begin{align}
        \label{eq:thl2_tilde_solved}
        \tilde{\theta}_{l2}
        &= -\frac{\widehat{c}_{w \theta_l}}{\widehat{w}_1}.
    \end{align}

    \item The widths of the normals, $\tilde{\sigma}_{\theta_l 1}$ and $\tilde{\sigma}_{\theta_l 2}$,
    are determined by satisfying \cref{eq:thlp2_bar} for $\thlptwo$,
    and \cref{eq:theta_l_3_bar} for $\thlpthree$:
    \begin{align}
        \label{eq:sigma_tilde_theta1_solved}
        \tilde{\sigma}_{\theta_l 1}^2
        &= \left(1-\widehat{c}_{w \theta_l}^2 \right) +
        \left(\sqrt{\frac{1 - \alpha}{\alpha}}\right)
        \frac{1}{3 \widehat{c}_{w \theta_l}}
        \left( \widehat{Sk_{\theta_l}} - \widehat{c}_{w \theta_l}^3 \widehat{Sk}_w \right),
    \end{align}
    \begin{align}
        \label{eq:sigma_tilde_theta2_solved}
        \tilde{\sigma}_{\theta_l 2}^2
        &= \left(1 - \widehat{c}_{w \theta_l}^2 \right) -
        \left(\sqrt{\frac{\alpha}{1 - \alpha}}\right)
        \frac{1}{3 \widehat{c}_{w \theta_l}}
        \left( \widehat{Sk_{\theta_l}} - \widehat{c}_{w \theta_l}^3 \widehat{Sk}_w \right).
    \end{align}
    Here $Sk_{\theta_l}$ is the skewness of $\theta_l$.
    It must be provided either by a prognostic equation or by a diagnostic equation
    such as \cref{eq:sk_hat_theta_l_nondim} below.

    \item Equations for $\tilde{r}_{t1}$, $\tilde{r}_{t2}$, $\tilde{\sigma}_{r_t 1}^2$,
    and $\tilde{\sigma}_{r_t 2}^2$ are found by expressions identical to \cref{eq:thl1_tilde_solved},
    \cref{eq:thl2_tilde_solved}, \cref{eq:sigma_tilde_theta1_solved},
    and \cref{eq:sigma_tilde_theta2_solved}, except that $\theta_l$ is replaced everywhere by $r_t$.

    \item Finally, from \cref{eq:r_t_prime_theta_l_prime_bar} for $\rtpthlp$ we find
    \begin{align}
        \label{eq:r_r_t_theta_l}
        r_{r_t \theta_l}
        &= \frac{\widehat{c}_{r_t \theta_l} - \widehat{c}_{w r_t} \widehat{c}_{w \theta_l}}
        {\alpha \tilde{\sigma}_{r_{t}1}\tilde{\sigma}_{\theta_{l}1} +
            (1-\alpha) \tilde{\sigma}_{r_{t}2} \tilde{\sigma}_{\theta_{l}2}}.
    \end{align}
    Here $r_{r_t \theta_l}$ is the in-between normal correlation and $c_{r_t \theta_l}$
    is the total correlation.
\end{enumerate}