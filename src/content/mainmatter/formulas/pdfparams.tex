\section{Solving for pdf parameters by using the moment terms}
\label{sec:solving-for-pdf-parameters-by-using-the-moment-terms}

Having established the prognosed moments for the desired \gls{pdf},
we now try to retrieve the specific \gls{pdf} that generates these moments.
This process essentially involves inverting the relationship between the moments
and the parameters that define the \gls{pdf}.

In our case, we refer back to the normal mixture family of \glspl{pdf} (\cref{eq:normal_mix_pdf}),
which offers a representation for atmospheric grid layers.
To select a particular member within this family that best aligns with the prognosed moments,
we perform a parameter retrieval step.

This retrieval is achieved by inverting equations~\eqref{eq:w_bar_nondim} to~\eqref{eq:c_hat_r_t_theta_l_nondim}.
These equations express the prognosed moments (mean, variance, covariances, etc.)
as functions of the underlying \gls{pdf} parameters (weights, means, and standard deviations).
By inverting these relationships,
we aim to find a set of \gls{pdf} parameters
that produces the distribution corresponding to the prognosed moments.

However, it is important to mention that this inversion is not a straightforward process.
That is because the equations are non-linear with respect to the \gls{pdf} parameters.
Despite this non-linearity,
the relatively simple structure of the normal mixture \gls{pdf} (\cref{eq:normal_mix_pdf})
allows for an analytical solution to the inversion problem.
This analytical solution enables us to efficiently map the prognosed moments
back to the corresponding \gls{pdf} parameters.

The proposed solution procedure~\cite{larson2005using} is as follows.
\begin{enumerate}
    \item Solve for $\alpha$, $\widehat{w}_1$,
    and $\widehat{w}_2$ from the equations for $\overline{w}$ (\cref{eq:w_bar_nondim}),
    $\wptwo$ (\cref{eq:wp2_bar_non_dim}), $\wpthree$ (\cref{eq:sk_hat_w_nondim}):
    \begin{align}
        \label{eq:alpha_solved}
        \alpha
        &= \frac{1}{2}\left[1 - \widehat{Sk}_w \sqrt{\frac{1}{4 + \widehat{Sk}_w^2}}\right],
    \end{align}
    \begin{align}
        \label{eq:w1_solved}
        \widehat{w}_1
        &= \sqrt{\frac{1-\alpha}{\alpha}},
    \end{align}
    \begin{align}
        \label{eq:w2_solved}
        \widehat{w}_2
        &= -\sqrt{\frac{\alpha}{1-\alpha}}.
    \end{align}
    Without loss of generality, it has been chosen to set $\widehat{w}_1 > \widehat{w}_2$.

    \item Looking at equation \cref{eq:alpha_solved},
    we see that $\widehat{Sk}_w$ is determined only by $\alpha$:
    \begin{align}
        \label{eq:sk_w_alpha}
        \widehat{Sk}_w
        &= \frac{1-2\alpha}{\sqrt{\alpha(1-\alpha)}}.
    \end{align}

    \item $\tilde{\theta}_{l1}$ and $\tilde{\theta}_{l2}$ are taken from solving \cref{eq:thlp_bar_nondim}
    for $\overline{\theta_l}$, and \cref{eq:c_hat_w_theta_l_nondim} for $\wpthlp$:
    \begin{align}
        \label{eq:thl1_tilde_solved}
        \tilde{\theta}_{l1}
        &= -\frac{\widehat{c}_{w \theta_l}}{\widehat{w}_2},
    \end{align}
    \begin{align}
        \label{eq:thl2_tilde_solved}
        \tilde{\theta}_{l2}
        &= -\frac{\widehat{c}_{w \theta_l}}{\widehat{w}_1}.
    \end{align}

    \item We can get $\tilde{\sigma}_{\theta_l 1}$ and $\tilde{\sigma}_{\theta_l 2}$ by fulfilling \cref{eq:thlp2_bar} for $\thlptwo$,
    and \cref{eq:theta_l_3_bar} for $\thlpthree$:
    \begin{align}
        \label{eq:sigma_tilde_theta1_solved}
        \tilde{\sigma}_{\theta_l 1}^2
        &= \left(1-\widehat{c}_{w \theta_l}^2 \right) +
        \left(\sqrt{\frac{1 - \alpha}{\alpha}}\right)
        \frac{1}{3 \widehat{c}_{w \theta_l}}
        \left( \widehat{Sk_{\theta_l}} - \widehat{c}_{w \theta_l}^3 \widehat{Sk}_w \right),
    \end{align}
    \begin{align}
        \label{eq:sigma_tilde_theta2_solved}
        \tilde{\sigma}_{\theta_l 2}^2
        &= \left(1 - \widehat{c}_{w \theta_l}^2 \right) -
        \left(\sqrt{\frac{\alpha}{1 - \alpha}}\right)
        \frac{1}{3 \widehat{c}_{w \theta_l}}
        \left( \widehat{Sk_{\theta_l}} - \widehat{c}_{w \theta_l}^3 \widehat{Sk}_w \right).
    \end{align}
    $Sk_{\theta_l}$ represents the skewness of $\theta_l$,
    which has to be provided by an equation such as \cref{eq:sk_hat_theta_l_nondim} below.

    \item Finding formulas for $\tilde{r}_{t1}$, $\tilde{r}_{t2}$, $\tilde{\sigma}_{r_t 1}^2$,
    and $\tilde{\sigma}_{r_t 2}^2$ can be done by replacing $\theta_l$ by $r_t$ everywhere
    in the equations~\eqref{eq:thl1_tilde_solved},
    ~\eqref{eq:thl2_tilde_solved},~\eqref{eq:sigma_tilde_theta1_solved},
    and~\eqref{eq:sigma_tilde_theta2_solved}.

    \item The last step is to get a relationship between $r_{r_t \theta_l}$,
    the in-between normal correlation and $c_{r_t \theta_l}$,
    the total correlation.
    This can be done by using \cref{eq:r_t_prime_theta_l_prime_bar}:
    \begin{align}
        \label{eq:r_r_t_theta_l}
        r_{r_t \theta_l}
        &= \frac{\widehat{c}_{r_t \theta_l} - \widehat{c}_{w r_t} \widehat{c}_{w \theta_l}}
        {\alpha \tilde{\sigma}_{r_{t}1}\tilde{\sigma}_{\theta_{l}1} +
            (1-\alpha) \tilde{\sigma}_{r_{t}2} \tilde{\sigma}_{\theta_{l}2}}.
    \end{align}
\end{enumerate}