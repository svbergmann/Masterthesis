\section{Third Normal}\label{sec:third_normal}

We would like to define a third Normal to the already existing two trivariate Normals which is right in the middle between those two.
Using the following formula for a multivariate Normal, we can just define the sum of our three Normals by using a vector for the mean, as well as the covariance matrix.
For the purpose of readability, we define the mean vectors of the first and second Normal distributions as $\mu_1 = (w_1, \theta_{l1}, r_{t1})^\top$, and $\mu_1 = (w_2, \theta_{l2}, r_{t2})^\top$, where $w_1 > w_2$ (due to a restriction in the code) and the covariance matrices as
\begin{align*}
    \Sigma_1 =
    \begin{pmatrix}
        \sigma_w^2 & 0                                                      & 0                                                      \\
        0          & \sigma_{\theta_{l1}}^2                                 & \rho_{\theta_l r_t} \sigma_{\theta_l 3} \sigma_{r_t 3} \\
        0          & \rho_{\theta_l r_t} \sigma_{\theta_l 3} \sigma_{r_t 3} & \sigma_{r_{t1}}^2
    \end{pmatrix},
    \text{ and }
    \Sigma_2 =
    \begin{pmatrix}
        \sigma_w^2 & 0                                                      & 0                                                      \\
        0          & \sigma_{\theta_{l2}}^2                                 & \rho_{\theta_l r_t} \sigma_{\theta_l 3} \sigma_{r_t 3} \\
        0          & \rho_{\theta_l r_t} \sigma_{\theta_l 3} \sigma_{r_t 3} & \sigma_{r_{t2}}^2.
    \end{pmatrix}
\end{align*}

It might be of interest that there is no correlation between $w$ and $\theta_l$ or $w$ and $r_t$.
That is to make the \glspl{pdf} mathematically more tractable by therefore also taking away some variability.
This is not the case for the third Normal though.
It has already been said, that we would like to place the third Normal right at the mean, therefore $\mu_3$ and $\Sigma_3$ are defined as:
\begin{align*}
    \mu_3 =
    \begin{pmatrix}
        \overline{w}        \\
        \overline{\theta_l} \\
        \overline{r_t}
    \end{pmatrix},
    \text{ and }
    \Sigma_3 =
    \begin{pmatrix}
        \sigma_{w 3}^2 &
        \rho_{w \theta_l 3} \sigma_{w 3} \sigma_{\theta_l 3} &
        \rho_{w r_t 3} \sigma_{w 3} \sigma_{r_t 3} \\
        \rho_{w \theta_l 3} \sigma_{w 3} \sigma_{\theta_l 3} &
        \sigma_{\theta_l 3}^2 &
        \rho_{\theta_l r_t 3} \sigma_{\theta_l 3} \sigma_{r_t 3} \\
        \rho_{w r_t 3} \sigma_{w 3} \sigma_{r_t 3} &
        \rho_{\theta_l r_t 3} \sigma_{\theta_l 3} \sigma_{r_t 3} &
        \sigma_{r_t 3}^2
    \end{pmatrix}.
\end{align*}
For our proposed mix of Normals we then have:
\begin{align}
    \label{eq:normal_mix_pdf}
    P_{tmg}(w, \theta_l, r_t)
    = \alpha (1-\delta) \mathcal{N}(\mu_1, \Sigma_1)
    + (1-\alpha) (1-\delta) \mathcal{N}(\mu_2, \Sigma_2)
    + \delta \mathcal{N}(\mu_3, \Sigma_3),
\end{align}
where $\mathcal{N}$ denotes the multivariate Normal Distribution.
The advantage over just two Normal \glspl{pdf} is that we can now express more shapes of functions.
We also define some additional relationships for this third Normal distribution:
\begin{align}
    \label{eq:lambda}
    \lambda_w = \frac{\sigma_{w 3}}{\wptwo}, \quad
    \lambda_\theta = \frac{\sigma_{\theta_l 3}}{\thlptwo}, \quad
    \lambda_r = \frac{\sigma_{r_t 3}}{\rtptwo},
\end{align}
\begin{align}
    \label{eq:lambda_two}
    \lambda_{\theta r} = \frac{\rho_{\theta_l r_t} \sigma_{\theta_l 3} \sigma_{r_t 3}}{\rtpthlp}, \quad
    \lambda_{w \theta} = \frac{\rho_{w \theta_l} \sigma_{w 3} \sigma_{\theta_l 3}}{\wpthlp}, \quad
    \lambda_{w r} = \frac{\rho_{w r_t} \sigma_{w 3} \sigma_{r_t 3}}{\wprtp}.
\end{align}
Hence, we can rewrite $\Sigma_3$ as
\begin{align}
    \Sigma_3 =
    \begin{pmatrix}
        \sigma_{w 3}^2 &
        \wpthlp \cdot \lambda_{w r} &
        \wprtp \cdot \lambda_{w r} \\
        \wpthlp \cdot \lambda_{w r} &
        \sigma_{\theta_l 3}^2 &
        \rtpthlp \cdot \lambda_{\theta r} \\
        \wprtp \cdot \lambda_{w r} &
        \rtpthlp \cdot \lambda_{\theta r} &
        \sigma_{r_t 3}^2
    \end{pmatrix}.
\end{align}