\section{Formulating closure relationships for higher-order moments}
\label{sec:prop_closure}

This section delves into the derivation of closure relationships for crucial higher-order moments
employed within the \gls{CLUBB} parameterization.

Our focus here lies on achieving closure for the following terms:
\begin{itemize}
    \item $\wpfour$: The fourth-order moment of up-/downdrafts,
    \item $\wptwothlp$: the soo called flux,
    \item $\wpthlptwo$: the counterpart of the previous one, where we replace every $\theta_l$ by $r_t$,
    \item $\wprtpthlp$: and the mean over all three standardized variables.
\end{itemize}
Closure, in this context,
refers to expressing these higher-order moments completely in terms of known quantities,
typically lower-order moments that are directly prognosed by the model.
The approach to derive those formulas is based on
the previously established expressions for the \gls{pdf} parameters (equations~\eqref{eq:alpha_solved} to~\eqref{eq:r_r_t_theta_l}).
By substituting those derived \gls{pdf} parameter expressions
into the relevant equations for the higher-order moments (equations~\eqref{eq:wp4_div_wp2_2} to~\eqref{eq:wprtpthlp_div_wp2_thp2_rt2}),
we find the desired closure relationships.

We first present the equation for the third moment of $\theta_l'$, $\overline{\theta_l'}$
This expression is derived by dimensionalizing \cref{eq:Sk_hat_thl_beta},
which relates the skewness of $\theta_l$ to the skewness of $w$ and other model parameters.
\begin{align}
    \label{eq:thlp3_beta}
    \overline{\theta_l'^3}
    &= \frac{(1-\delta \lambda_{w \theta})(1-\delta \lambda_\theta)}{\tswfact^2 (1-\delta \lambda_w)^2}
    \frac{\overline{w'^3}}{\left( \overline{w'^2} \right)^2}
    \overline{\theta_l'^2} \;
    \overline{w'\theta_l'}
    \left(\beta + (1-\beta)
    \frac{(1-\delta \lambda_{w \theta})^2}{1-\tilde{\sigma}_w^2 (1-\delta \lambda_w)(1-\delta \lambda_\theta)}
    \frac{\left(\overline{w'\theta_l'}\right)^2}{\overline{w'^2} \; \overline{\theta_l'^2}}\right).
\end{align}
While the scalar third moments (e.g., $\thlpthree$) may not directly participate
in solving the prognostic equations within \gls{CLUBB},
they hold an indirect yet crucial role in shaping cloud properties within atmospheric simulations.
This influence is coming from the connection between the \gls{pdf} and the cloud formation.

Cumulus cloud formation mostly occurs at the edges,
or \enquote{tails} of the \gls{pdf} for a specific variable.
These tails represent regions where the probability of encountering extreme values of the variable is relatively higher.
As the relative \enquote{width} of the normal distribution representing $w$ increases,
the magnitude of $\thlpthree$ also grows (refer to ~\eqref{eq:thlp3_beta} for details).
In simpler terms, a larger value of $\thlpthree$ corresponds to a broader \gls{pdf} for the up-/downdraft variable.
This broader \gls{pdf} deviates more significantly from a double delta function,
which is a construct with two spikes at zero.

Unlike the scalar third moment,
$\wpfour$ does not depend on the thermodynamic scalar moments (such as $\thlpthree$).
Consequently, it is independent of the parameter $\beta$.

To derive the explicit formula for $\wpfour$,
we can substitute the previously stated expressions for $\widehat{w_1}$ (\cref{eq:w1_solved})
and $\widehat{w_2}$ (\cref{eq:w2_solved}) into \cref{eq:wp4_div_wp2_2}.
\begin{align}
    \label{eq:wp4_div_wp2_2_solved}
    \frac{1}{\tswfact^2} \frac{(1-\delta)}{(1-\delta \lambda_w)^2} \frac{\wpfour}{\left(\wptwo\right)^2}
    &= 3 \frac{\tsw^4}{\tswfact^2} + 6 \frac{\tsw^2}{\tswfact} + 1 \nonumber\\
    &+ \widehat{Sk}_w^2 + \frac{1}{\tswfact^2} \frac{(1-\delta)}{(1-\delta \lambda_w)^2} \delta 3 \lambda_w^2,
\end{align}
and also
\begin{align}
    \label{eq:wp4}
    \overline{w'^4}
    &= \left(\overline{w'^2}\right)^2
    \frac{(1-\delta \lambda_w)^2}{(1-\delta)}
    \left(3 \tilde{\sigma}_w^4 + 6 \tswfact \tilde{\sigma}_w^2 + \tswfact^2\right) \nonumber\\
    &+ \frac{1}{\tswfact} \frac{1}{(1-\delta \lambda_w)}
    \frac{\left(\overline{w'^3}\right)^2}{\overline{w'^2}} \nonumber\\
    &+ \delta 3 \lambda_w^2 \left(\overline{w'^2}\right)^2.
\end{align}
As observed with $\wpfour$,
$\wptwothlp$ displays independence from the parameter $\beta$
within the context of the chosen \gls{pdf}.

To proceed, we can substitute the previously derived expressions for
$\widehat{w_1}$ (\cref{eq:w1_solved}), $\widehat{w_2}$ (\cref{eq:w2_solved}),
$\widehat{Sk}_w$ (\cref{eq:sk_hat_w_nondim}),
$\widehat{\tilde{\theta}_{l1}}$ (\cref{eq:thl1_tilde_solved}),
and $\widehat{\tilde{\theta}_{l2}}$ (\cref{eq:thl2_tilde_solved}) into \cref{eq:wp2thlp_div_wp2thl2}.
This substitution process will yield an explicit formula for $\wptwothlp$ that solely relies on known quantities,
such as the prognostic moments directly calculated by the model.
\begin{align}
    \label{eq:wp2thlp_sk_w_solved}
    \frac{1}{\tswfact} \frac{(1-\delta)^{1/2}}{(1-\delta \lambda_w) (1-\delta \lambda_\theta)^{1/2}} \frac{\overline{w'^2\theta_l'}}{\overline{w'^2} \left(\overline{\theta_l'^2} \right)^{1/2}}
    &= \widehat{c}_{w\theta_l} \widehat{Sk}_w,
\end{align}
and
\begin{align}
    \label{eq:wp2thlp_solved}
    \overline{w'^2\theta_l'}
    &= \frac{1}{\tswfact} \frac{1 - \delta \lambda_{w\theta}}{1-\delta \lambda_w} \frac{\overline{w'^3}}{\overline{w'^2}} \overline{w'\theta_l'}.
\end{align}
$\wpthlptwo$ depends explicitly on $Sk_{\theta_l}$.
Substituting \cref{eq:w1_solved} - \cref{eq:sigma_tilde_theta2_solved}
into \cref{eq:wprtpthlp_div_wp2_thp2_rt2} yields
\begin{align}
    \label{eq:wpthlp2_sk_w_solved}
    \frac{1}{\tswfact^{1/2}} \frac{(1-\delta)^{1/2}}{(1-\delta \lambda_w)^{1/2} (1-\delta \lambda_\theta)} \frac{\overline{w'\theta_l'^2}}{\left(\overline{w'^2}\right)^{1/2} \overline{\theta_l'^2}}
    &= \frac{2}{3} \widehat{c}_{w\theta_l}^2 \widehat{Sk}_w + \frac{1}{3} \frac{ \widehat{Sk_{\theta_l}} } {\widehat{c}_{w\theta_l}},
\end{align}
and
\begin{align}
    \label{eq:wpthlp2_solved}
    \overline{w'\theta_l'^2}
    &= \frac{2}{3} \frac{(1-\delta \lambda_{w \theta})^2}{(1-\delta \lambda_w)^2} \frac{1}{\tswfact^2} \frac{\overline{w'^3}}{\left(\overline{w'^2} \right)^2} \left(\overline{w'\theta_l'} \right)^2 \nonumber\\
    &+ \frac{1}{3} \frac{(1-\delta \lambda_w)}{(1-\delta \lambda_{w \theta})} \tswfact \frac{\overline{w'^2} \; \overline{\theta_l'^3} }{\overline{w'\theta_l'}}.
\end{align}

\textbf{\textcolor{red}{PROCEED HERE!}}

The formula has a problem because $\wpthlp$ is in the denominator.
As $\wpthlp$ gets closer to zero, the formula becomes infinitely large, which is called a singularity.
This can cause issues if we use the formula directly with real-world measurements of $\thlpthree$ and $\wpthlp$.
The resulting diagnosis of $\wpthlptwo$ would be very sensitive to small changes in the measurements
and might not be reliable (noisy).
We can fix this singularity by either
\begin{itemize}
    \item substitute in the ansatz for $Sk_{\theta_l}$ (\cref{eq:sk_hat_theta_l_nondim})
    into the original formula (\cref{eq:wpthlp2_div_wp2_thlp2}),
    \item or, equivalently, substitute \cref{eq:thlp3_beta} for $\thlpthree$.
    This is possible because \cref{eq:thlp3_beta} shows that $\thlpthree$ is proportional to $\wpthlp$.
\end{itemize}
Both approaches effectively remove the singularity from the formula.
Therefore, we find:
\begin{align}
    \label{eq:wpthlp2_div_wp2_thlp2_sk_w}
    \frac{1}{\tswfact^{1/2}} \frac{(1-\delta)^{1/2}}{(1-\delta \lambda_w)^{1/2} (1-\delta \lambda_\theta)} \frac{\overline{w'\theta_l'^2}}{\left(\overline{w'^2}\right)^{1/2} \overline{\theta_l'^2}}
    &= \widehat{Sk}_w \left[\frac{1}{3} \beta + \left(1 - \frac{1}{3} \beta \right) \widehat{c}_{w\theta_l}^2 \right],
\end{align}
and
\begin{align}
    \label{eq:wpthlp2_beta}
    \overline{w'\theta_l'^2}
    &= \frac{1}{\tswfact} \frac{(1-\delta \lambda_\theta)}{(1-\delta \lambda_w)}
    \frac{\overline{w'^3}}{\overline{w'^2}}
    \left[ \frac{1}{3} \beta \overline{\theta_l'^2}
    + \frac{\left( 1-\frac{1}{3} \beta \right)}{\tswfact}
    \frac{(1-\delta \lambda_{w \theta})^2}{(1-\delta \lambda_w)(1-\delta \lambda_\theta)}
    \frac{\left( \overline{w'\theta_l'} \right)^2}{\overline{w'^2}}
    \right].
\end{align}
Finally, substituting \cref{eq:w1_solved} - \cref{eq:r_r_t_theta_l} into \cref{eq:wprtpthlp_div_wp2_thp2_rt2}
yields the following formula for the turbulent flux of $\rtpthlp$, $\wprtpthlp$:
\begin{align}
    \label{eq:wprtpthlp_div_wp2_thp2_rt2_sk_w}
    \frac{(1-\delta)^{1/2}}{(1-\delta \lambda_w)^{1/2}
        (1-\delta \lambda_\theta)^{1/2}
        (1-\delta \lambda_{r_t})^{1/2}}
    \frac{\wprtpthlp}{\tswfact^{1/2}
        \left( \overline{w'^2} \right)^{1/2}
        \left(\rtptwo\right)^{1/2} \left(\thlptwo\right)^{1/2}} \nonumber \\
    = \widehat{c}_{wr_t} \widehat{c}_{w\theta_l} \widehat{Sk}_w
    + E(w,q_t,\theta_l) \frac{1}{2} \widehat{Sk}_w
    \left( c_{q_t\theta_l} - \widehat{c}_{wr_t} \widehat{c}_{w\theta_l}\right),
\end{align}
and
\begin{align}
    \label{eq:wprtpthlp_E}
    \wprtpthlp
    &= \frac{\frac{1}{2} E}{\tswfact}
    \frac{(1-\delta \lambda_{\theta q})} {(1 - \delta \lambda_w)}
    \rtpthlp \frac{\overline{w'^3}}{\overline{w'^2}} \nonumber\\
    &+ \frac{1 - \frac{1}{2} E}{\tswfact^2}
    \frac{(1-\delta \lambda_{w q})(1-\delta \lambda_{w \theta})}{(1 - \delta \lambda_w)^2}
    \wprtp \; \wpthlp \frac{\overline{w'^3}}{\left( \overline{w'^2} \right)^2}.
\end{align}
The function $E(w,r_t,\theta_l)$ is
\begin{align}
    \label{eq:E}
    E &= \frac{1 - \frac{1}{2} \frac{2\alpha}{1-2\alpha} \xi}{1 + \frac{1}{2} \xi},
\end{align}
where
\begin{align}
    1 + \xi
    = \frac{1-\alpha}{\alpha} \frac{\tilde{\sigma}_{r_{t2}}}{\tilde{\sigma}_{r_{t1}}} \frac{ \tilde{\sigma}_{\theta_{l2}}}{\tilde{\sigma}_{\theta_{l1}}}
    = \left(\frac{A_{r_t} - B_{r_t}}{-A_{r_t} - B_{r_t}}\right)^{1/2} \left(\frac{A_{\theta_l} - B_{\theta_l}}{-A_{\theta_l} - B_{\theta_l}}\right)^{1/2},
\end{align}
and
\begin{align}
    \label{eq:A_thl}
    A_{\theta_l}
    &= Sk_{\theta_l} - \frac{3}{2} \widehat{c}_{w\theta_l} \widehat{Sk}_w + \frac{1}{2} \widehat{c}_{w\theta_l}^3 \widehat{Sk}_w,
\end{align}
\begin{align}
    B_{\theta_l}
    &= \frac{3}{2} \left(4 + \widehat{Sk}_w^2 \right)^{1/2} \widehat{c}_{w\theta_l} \left(1 - \widehat{c}_{w\theta_l}^2\right),
\end{align}
and $A_{r_t}$ and $B_{r_t}$ are analogous.
We now list two cases in which the expression for $E$ simplifies.
First, if
\begin{align*}
    \frac{\tilde{\sigma}_{r_{t2}}}{\tilde{\sigma}_{r_{t1}}} \frac{\tilde{\sigma}_{\theta_{l2}}}{\tilde{\sigma}_{\theta_{l1}}}
    &= 1
\end{align*}
then $E=0$.
This would occur, for instance,
if the \enquote{widths} of the first and second normal were equal to each other for both $r_t$ and $\theta_l$,
that is, if $\tilde{\sigma}_{r_{t2}} = \tilde{\sigma}_{r_{t1}}$
and $\tilde{\sigma}_{\theta_{l2}} = \tilde{\sigma}_{\theta_{l1}}$.
Second, if we use the diagnostic ansatz (\cref{eq:sk_hat_theta_l_nondim}) for the scalar skewnesses, then
\begin{align}
    \label{eq:xi}
    \xi = \frac{1 - 2 \zeta}{\zeta},
\end{align}
where
\begin{align}
    \label{eq:zeta}
    \zeta = \alpha + \frac{1}{3} \beta \left(1 - 2 \alpha \right).
\end{align}

Then we find
\begin{align}
    \label{eq:E_eq_23beta}
    E &= \frac{2}{3} \beta,
\end{align}
and finally
\begin{align}
    \label{eq:wpqtpthlp_beta}
    \wprtpthlp
    &= \frac{\frac{1}{3}\beta}{\tswfact}
    \frac{(1-\delta \lambda_{\theta r})}{(1 - \delta \lambda_w)}
    \rtpthlp \frac{\overline{w'^3}}{\overline{w'^2}} \nonumber\\
    &+ \frac{1 - \frac{1}{3}\beta}{\tswfact^2}
    \frac{(1-\delta \lambda_{w r})(1-\delta \lambda_{w \theta})} {(1 - \delta \lambda_w)^2}
    \wprtp \; \wpthlp \frac{\overline{w'^3}}{\left( \overline{w'^2} \right)^2}.
\end{align}