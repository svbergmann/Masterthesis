\section{Deduced Moments}\label{sec:dedmoments}

We start by outlining the equations capturing lower-order moments,
expressed in terms of \gls{pdf} parameters.
These equations can be presented in either \emph{dimensional} or \emph{non-dimensional} form.
While both representations are mathematically valid,
the \emph{non-dimensional} form offers a distinct advantage:
it highlights the underlying connection to the bivariate case.

\subsection{Lower Order Moments}\label{subsec:lowerordermoments}

\subsubsection{Moments for $w$}\label{subsubsec:lowerordermoments_w}

The relationship for $\overline{w}$ is given as follows:
\begin{align}
    \label{eq:w_bar}
    \overline{w} = (1 - \delta) \alpha w_1 + (1 - \delta)(1-\alpha) w_2 + \delta w_3,
\end{align}
where $w_3 \equiv \alpha w_1 + (1 - \alpha) w_2$.
Therefore the mean of $w$ stays the same as in the bivariate case.
The relationship for the \emph{non-dimensional} form is:
\begin{align}
    \label{eq:w_bar_nondim}
    0 &= \alpha \widehat{w}_1 + (1 - \alpha) \widehat{w}_2
\end{align}

For all other moments -- except for the mean --
we are using the standardized versions of the variables, written as \gls{w_prime}.

The second order moment is given as:
\begin{align}
    \label{eq:wp2_bar}
    \wptwo
    &= (1 - \delta) \alpha [(w_1 - \overline{w})^2 + \sigma_w^2] \nonumber\\
    &+ (1 - \delta) (1 - \alpha) [(w_2 - \overline{w})^2 + \sigma_w^2] \nonumber\\
    &+ \delta \sigma_{w 3}^2,
\end{align}
where $\sigma_{w 3}$ is defined as $\lambda_w \wptwo$.
This moment is also the variance of $w$ at the same time, since
\begin{align*}
    \overline{w'^2}
    &= \mathbb{E}[w'^2]
    = \mathbb{E}[(w - \overline{w})^2]
    = \mathbb{E}[(w - \mathbb{E}[w])^2]
    = \mathbb{E}[w^2 -2w\mathbb{E}[w] + \mathbb{E}[w]^2] \\
    &= \mathbb{E}[w^2] - 2\mathbb{E}[w\mathbb{E}[w]] + \mathbb{E}[\mathbb{E}[w]^2]
    = \mathbb{E}[w^2] - 2\mathbb{E}[w]\mathbb{E}[w] + \mathbb{E}[w]^2 \\
    &= \mathbb{E}[w^2] - \mathbb{E}[w]^2
    = \text{Var}[w].
\end{align*}
The \emph{non-dimensional} relationship would then be:
\begin{align}
    \label{eq:wp2_bar_non_dim}
    \frac{1}{\tswfact}
    &= \alpha \left(\widehat{w}_1^2 + \frac{\tilde{\sigma_w}^2}{(1 - \tilde{\sigma_w})^2} \right)
    + (1-\alpha) \left( \widehat{w}_2^2 + \frac{\tsw^2}{\tswfact} \right).
\end{align}

The third order moment is given as:
\begin{align}
    \label{eq:wp3_bar}
    \wpthree
    &= (1 - \delta) \alpha [(w_1 - \overline{w})^3 + 3 \sigma_w^2 (w_1 - \overline{w})] \nonumber\\
    &+ (1 - \delta) (1 - \alpha) [(w_2 - \overline{w})^3 + 3 \sigma_w^2 (w_2 - \overline{w})]
\end{align}

Since we want to make use of the specific shape of the \gls{pdf},
we also have a relationship for $w'^3$,
which is called $\widehat{Sk}_w$, meaning the skewness of the variable $w$:
\begin{align}
    \widehat{Sk}_w
    &\equiv \frac{1}{\tswfact^{3/2}} \frac{\wpthree}{\left( \wptwo \right)^{3/2}}\frac{1}{\left(\frac{1-\delta \lambda_w}{1-\delta}\right)^{3/2}}\frac{1}{1-\delta} \nonumber \\
    &= \alpha \left(\widehat{w}_1^3 + 3 \widehat{w}_1 \frac{\tsw^2}{\tswfact} \right) + (1 - \alpha) \left( \widehat{w}_2^3 + 3 \widehat{w}_2 \frac{\tsw^2}{\tswfact} \right)
    \label{eq:sk_hat_w_nondim}
\end{align}

\subsubsection{Moments for $\theta_l$}\label{subsubsec:lowerordermoments_thl}

For $\theta_l$ we have a similar \emph{non-dimensional} relationship:
\begin{align}
    \label{eq:thlp_bar_nondim}
    0 &= \alpha \tilde{\theta}_{l1} + (1 - \alpha) \tilde{\theta}_{l2}
\end{align}
Similarly but with a different standard deviation, $\thlptwo$ is given as:
\begin{align}
    \label{eq:thlp2_bar}
    \thlptwo
    &= (1 - \delta) \alpha [(\theta_{l1} - \overline{\theta_l})^2 + \sigma_{\theta_{l1}}^2] \nonumber\\
    &+ (1 - \delta) (1 - \alpha) [(\theta_{l2} - \overline{\theta_l})^2 + \sigma_{\theta_{l2}}^2] \nonumber\\
    &+ \delta \sigma_{\theta_l 3}^2,
\end{align}
where $\sigma_{\theta_l 3}$ is defined as $\lambda_{\theta_l} \thlptwo$.
This can also be expressed as the variance following the same approach as the one for $\wptwo$.

The third order moment is given as:
\begin{align}
    \label{eq:theta_l_3_bar}
    \thlpthree
    &= (1 - \delta) \alpha [(\theta_{l1} - \overline{\theta_l})^3
    + 3 \sigma_{\theta_{l1}}^2 (\theta_{l1} - \overline{\theta_l})] \nonumber\\
    &+ (1 - \delta) (1 - \alpha) [(\theta_{l2} - \overline{\theta_l})^3
    + 3 \sigma_{\theta_{l2}}^2 (\theta_{l2} - \overline{\theta_l})]
\end{align}
Similarly to \cref{eq:sk_hat_w_nondim}, we also list a moment which is more diagnosed than prognosed:
\begin{align}
    \widehat{Sk_{\theta_l}}
    &\equiv \frac{\thlpthree}{\left( \thlptwo \right)^{3/2}}\left(
    \frac{1}{\frac{1-\delta \lambda_\theta}{1-\delta}}
    \right)^{3/2}
    \frac{1}{1-\delta} \nonumber\\
    &= \alpha \left( \tilde{\theta}_{l1}^3 + 3 \tilde{\theta}_{l1} \tilde{\sigma}_{\theta_l 1}^2 \right)
    + (1 - \alpha) \left( \tilde{\theta}_{l2}^3 + 3 \tilde{\theta}_{l2} \tilde{\sigma}_{\theta_l 2}^2 \right).
    \label{eq:sk_hat_theta_l_nondim}
\end{align}

\subsubsection{Moments for $r_t$}\label{subsubsec:lowerordermoments_rt}

The relationships for $r_t$ and $r_t'^2$ are given as follows
\begin{align}
    \label{eq:rt_bar_nondim}
    0 &= \alpha \tilde{r}_{t1} + (1 - \alpha) \tilde{r}_{t2},
\end{align}
and
\begin{align}
    \label{rtptwo_nondim}
    1 &= \alpha \left( \tilde{r}_{t1}^2 + \tilde{\sigma}_{r_{t1}}^2 \right) + (1 - \alpha) \left( \tilde{r}_{t2}^2 + \tilde{\sigma}_{r_{t2}}^2 \right).
\end{align}
Since this relationship is similar to the relationships of $\theta_l$ and $\theta_l'^2$,
we are using kind of the same formulas:
\begin{align}
    \label{eq:r_t_prime_2_bar}
    \rtptwo
    &= (1 - \delta) \alpha [(r_{t1} - \overline{r_t})^2 + \sigma_{r_{t1}}^2] \nonumber\\
    &+ (1 - \delta) (1 - \alpha) [(r_{t2} - \overline{r_t})^2 + \sigma_{\theta_{l2}}^2] \nonumber\\
    &+ \delta \sigma_{r_t 3}^2,
\end{align}
and
\begin{align}
    \label{eq:r_t_prime_3_bar}
    \rtpthree
    &= (1 - \delta) \alpha [(r_{t1} - \overline{r_t})^3 + 3 \sigma_{r_{t1}}^2 (r_{t1} - \overline{r_t})] \nonumber\\
    &+ (1 - \delta) (1 - \alpha) [(r_{t2} - \overline{r_t})^3 + 3 \sigma_{r_{t2}}^2 (r_{t2} - \overline{r_t})]
\end{align}

\subsubsection{Mixed Moments}\label{subsubsec:lowerordermoments_mixed}

There are also equations for two or even three variables, which are listed in the following.
\begin{align}
    \label{eq:w_prime_theta_l_prime_bar}
    \wpthlp
    &= (1 - \delta) \alpha [(w_1 - \overline{w}) (\theta_{l1} - \overline{\theta_l})] \nonumber\\
    &+ (1 - \delta) (1 - \alpha) [(w_2 - \overline{w}) (\theta_{l2} - \overline{\theta_l})] \nonumber\\
    &+ \delta \lambda_{w\theta} \wpthlp,
\end{align}
\begin{align}
    \label{eq:w_prime_r_t_prime_bar}
    \wprtp
    &= (1 - \delta) \alpha [(w_1 - \overline{w}) (r_{t1} - \overline{r_t})] \nonumber\\
    &+ (1 - \delta) (1 - \alpha) [(w_2 - \overline{w}) (r_{t2} - \overline{r_t})] \nonumber\\
    &+ \delta \lambda_{wr} \wprtp,
\end{align}
and
\begin{align}
    \label{eq:r_t_prime_theta_l_prime_bar}
    \rtpthlp
    &= (1 - \delta) \alpha [(r_{t1} - \overline{r_t}) (\theta_{l1} - \overline{\theta_l}) + r_{r_t \theta_l} \sigma_{r_{t1}} \sigma_{\theta_{l1}}] \nonumber\\
    &+ (1 - \delta) (1 - \alpha) [(r_{t2} - \overline{r_t}) (\theta_{l2} - \overline{\theta_l}) + r_{r_t \theta_l} \sigma_{r_{t2}} \sigma_{\theta_{l2}}] \nonumber\\
    &+ \delta \lambda_{r\theta} \overline{r_t' \theta_l'}.
\end{align}
We have the \emph{non-dimensional} relationship for those moments given as
\begin{align}
    \widehat{c}_{w \theta_l}
    &\equiv \frac{1}{\tswfact^{1/2}} \frac{\overline{w' \theta_l'}}{\sqrt{\wptwo}\sqrt{\thlptwo}} \frac{1}{\sqrt{\frac{1 - \delta \lambda_w}{1 - \delta}}}\frac{1}{\sqrt{\frac{1 - \delta \lambda_\theta}{1 - \delta}}} \frac{1 - \delta \lambda_{w\theta}}{1 - \delta} \nonumber\\
    &= \alpha \widehat{w}_1 \tilde{\theta}_{l1} + (1 - \alpha) \widehat{w}_2 \tilde{\theta}_{l2},
    \label{eq:c_hat_w_theta_l_nondim}
\end{align}
\begin{align}
    \widehat{c}_{w r_t}
    &\equiv \frac{1}{\tswfact^{1/2}} \frac{\overline{w' r_t'}}{\sqrt{\wptwo}\sqrt{\rtptwo}} \frac{1}{\sqrt{\frac{1 - \delta \lambda_w}{1 - \delta}}} \frac{1}{\sqrt{\frac{1 - \delta \lambda_r}{1-\delta}}} \frac{1 - \delta \lambda_{w r}}{1-\delta} \nonumber\\
    &= \alpha \widehat{w}_1 \tilde{r}_{t1} + (1 - \alpha) \widehat{w}_2 \tilde{r}_{t2},
    \label{eq:c_hat_w_r_t_nondim}
\end{align}
and
\begin{align}
    \widehat{c}_{r_t \theta_l}
    &\equiv\frac{\rtpthlp}{\sqrt{\rtptwo}\sqrt{\thlptwo}} \frac{1}{\sqrt{\frac{1-\delta \lambda_q}{1-\delta}}}\frac{1}{\sqrt{\frac{1-\delta \lambda_\theta}{1-\delta}}}\frac{1 - \delta \lambda_{\theta r} }{1 - \delta} \nonumber\\
    &= \alpha \left( \tilde{r}_{t1} \tilde{\theta}_{l1} + r_{r_t \theta_l} \tilde{\sigma}_{q_{t1}} \tilde{\sigma}_{\theta_{l1}} \right) + (1 - \alpha) \left( \tilde{r}_{t2} \tilde{\theta}_{l2} + r_{r_t \theta_l} \tilde{\sigma}_{r_{t2}} \tilde{\sigma}_{\theta_{l2}} \right),
    \label{eq:c_hat_r_t_theta_l_nondim}
\end{align}
where we can think about $\widehat{c}$ as the correlation.

We also list a trivariate moment ($\wprtpthlp$), given by:
\begin{align}
    \wprtpthlp
    &= (1 - \delta) \alpha (w_1 - \overline{w}) [(r_{t1} - \overline{r_t}) (\theta_{l1} - \overline{\theta_l}) + r_{r_t \theta_l} \sigma_{r_{t1}} \sigma_{\theta_{l1}}] \nonumber\\
    &+ (1 - \delta) (1 - \alpha) (w_2 - \overline{w}) [(r_{t2} - \overline{r_t}) (\theta_{l2} - \overline{\theta_l}) + r_{r_t \theta_l} \sigma_{r_{t2}} \sigma_{\theta_{l2}}]
    \label{eq:w_prime_r_t_prime_t_l_prime_bar}
\end{align}