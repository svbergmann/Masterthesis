\section{A diagnostic ansatz for the skewness of heat and moisture}\label{sec:diag_ansatz}

We cannot close the system of equations until we specify the skewness of $\theta_l$, $Sk_{\theta_l}$,
that appears in \cref{eq:sigma_tilde_theta1_solved} for $\tilde{\sigma}_{\theta_l 1}$,
and \cref{eq:sigma_tilde_theta2_solved} for $\tilde{\sigma}_{\theta_l 2}$.
Likewise, we need to specify $Sk_{r_t}$.
We could prognose these scalar skewnesses, but this would involve additional computational expense,
storage, and complexity.
In some cases, the extra complexity may be worthwhile.
However, here we instead propose the following diagnostic formula.
\begin{align}
    \label{eq:Sk_hat_thl_beta}
    \widehat{Sk}_{\theta_l}
    &= \widehat{Sk}_w \widehat{c}_{w \theta_l} \left[\beta + (1-\beta) \widehat{c}_{w \theta_l}^2 \right],
\end{align}
and a similar formula for $\widehat{Sk}_{r_t}$.
The parameter $\beta$ is dimensionless.
We also solve for $\beta$ because we are going to need the equation later on to show that other equations are true.
\begin{align}
    \label{eq:beta}
    \implies \beta
    &=\frac{
        \frac{\widehat{Sk}_{\theta_l}}{\widehat{Sk}_w \widehat{c}_{w \theta_l}} - \widehat{c}_{w \theta_l}^2}
    {1 - \widehat{c}_{w \theta_l}^2}
\end{align}
\Cref{eq:Sk_hat_thl_beta} is physically plausible but limited at the same time.
The formula states that $Sk_{\theta_l}$ is proportional to $Sk_w$, the skewness of $w$.
An increase in $\beta$ leads to an increase in $\left| Sk_{\theta_l} \right|$, which, in turn,
leads to a \gls{pdf} with a longer $\theta_l$-tail.
$Sk_{\theta_l}$ and $Sk_w$ have the same sign when $w$ and $\theta_l$ are positively correlated;
$Sk_{\theta_l}$ and $Sk_w$ have opposite sign when $w$ and $\theta_l$ are negatively correlated.
In those large eddy simulations, this is usually but not always true.
$Sk_{\theta_l}$ vanishes when either $Sk_w$ or $c_{w \theta_l}$ vanishes;
clearly this need not be true in nature.
$\left| Sk_{\theta_l} \right|$ can be either smaller or larger than $\left| Sk_w \right|$,
depending on the values of $\tsw^2$, $c_{w \theta_l}$, and $\beta$.

If we assume our ansatz (\cref{eq:sk_hat_theta_l_nondim}) for $\widehat{Sk}_{\theta_l}$,
then the $\theta_l$-widths of the first (\cref{eq:sigma_tilde_theta1_solved})
and second normal (\cref{eq:sigma_tilde_theta1_solved}) reduce to
\begin{align}
    \label{eq:sigma_theta1_beta}
    \tilde{\sigma}_{\theta_l 1}^2
    &= \frac{\left(1 - \widehat{c}_{w \theta_l}^2\right)}{\alpha} \left[\frac{1}{3} \beta + \alpha \left(1 - \frac{2}{3} \beta\right)\right],
\end{align}
and
\begin{align}
    \label{eq:sigma_theta2_beta}
    \tilde{\sigma}_{\theta_l 2}^2
    &= \frac{\left(1 - \widehat{c}_{w \theta_l}^2\right)} {1 - \alpha} \left\{1 - \left[\frac{1}{3}\beta + \alpha \left(1 - \frac{2}{3} \beta \right)\right]\right\}.
\end{align}
Substituting \cref{eq:sigma_theta1_beta}, \cref{eq:sigma_theta2_beta},
and their $r_t$ counterparts into the expression for $r_{r_t \theta_l}$ \cref{eq:r_r_t_theta_l}
yields the simplified form:
\begin{align}
    \label{eq:r_r_t_theta_l_beta}
    r_{r_t \theta_l}
    &= \frac{c_{r_t \theta_l} - \widehat{c}_{w r_t} \widehat{c}_{w \theta_l}}{\left(1 - \widehat{c}_{w r_t}^2\right)^{1/2} \left(1 - \widehat{c}_{w \theta_l}^2\right)^{1/2}}.
\end{align}
Here, $r_{r_t \theta_l}$ is the correlation of $r_t$ and $\theta_l$ in-between the normals
and $c_{r_t \theta_l}$ is the total correlation.

\gls{CLUBB} chose a specific formula for the $w$-\enquote{width} of the individual normals,
which is the following.
\begin{align}
    \label{eq:sigma_tilde_w_restricted}
    \tsw^2
    &= \gamma \left[1 - \max(c_{w \theta_l}^2, c_{w r_t}^2) \right].
\end{align}
This makes $\tsw^2$ depend on $c_{w r_t}^2$ and $c_{w \theta_l}^2$.
Here $0 \leq \gamma < 1$ is a dimensionless constant.
This formula helps ensure that when $c_{w r_t}$ or $c_{w \theta_l}$ becomes large in magnitude,
$0 \leq \widehat{c}_{w \theta_l}^2, \widehat{c}_{w r_t}^2 < 1$
and hence $\tilde{\sigma}_{r_t 1,2}^2$, $\tilde{\sigma}_{\theta_l 1,2}^2$, and $r_{r_t \theta_l}$ remain realistic.