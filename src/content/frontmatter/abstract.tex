\begin{center}
    ABSTRACT
    \\
    \singlespacing
    \mytitle\\
    \doublespacing
    by\\
    \myauthor\\
    \singlespacing
    The University of Wisconsin--Milwaukee, 2024\\
    Under the Supervision of Professor Vince Larson
\end{center}

The \gls{CLUBB} model uses the sum of two normal \gls{pdf} components to represent subgrid variability
within a single grid layer of an atmospheric model.
This binormal approach, while computationally efficient,
restricts the model's ability to capture the full spectrum of potential shapes encountered in real-world atmospheric data.

This thesis proposes to introduce a third normal \gls{pdf} component strategically positioned between the existing two,
significantly enhancing the model's representational flexibility.
This trinormal representation allows for a wider range of grid-layer shapes
while permitting analytic solutions for certain higher order moments.

The core of this work lies in deriving the necessary mathematical transformations
for incorporating the third normal \gls{pdf} seamlessly into the \gls{CLUBB} framework.
This thesis lists all formulas, inputs, and outputs associated with the extended model
as well as gives an outline on how to check those equations.
Additionally,
it describes certain asymptotic behavior of the trinormal \gls{pdf}
under various parameter settings.
