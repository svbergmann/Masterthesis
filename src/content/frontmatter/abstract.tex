\begin{center}
    ABSTRACT
    \\
    \singlespacing
    ADDING A THIRD GAUSSIAN TO CLUBB\\
    \doublespacing
    by\\
    Sven Bergmann\\
    \singlespacing
    The University of Wisconsin--Milwaukee, 2024\\
    Under the Supervision of Professor Vince Larson
\end{center}

The \gls{CLUBB} model plays a crucial role in simulating atmospheric phenomena.
It uses the sum of two normal \glspl{pdf} to represent subgrid variability within a single grid layer.
This binormal approach, while computationally efficient,
restricts the model's ability to capture the full spectrum of potential shapes encountered in real-world atmospheric data.

This thesis proposes an innovative extension to the \gls{CLUBB} model.
We introduce a third normal \gls{pdf} strategically positioned between the existing two,
significantly enhancing the model's representational flexibility.
This trinormal representation allows for a wider range of grid-layer shapes
while permitting analytic solutions for certain higher order moments.

The core of this work lies in deriving the necessary mathematical transformations
for incorporating the third normal \gls{pdf} seamlessly into the \gls{CLUBB} framework.
This thesis lists all formulas, inputs, and outputs associated with the extended model.
Additionally,
it tries to describe certain asymptotic behavior of the trinormal \gls{pdf}
under various parameter settings.
