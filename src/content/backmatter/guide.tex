\section{User Guide}\label{sec:userguide}

\subsection{Accessing the Code}\label{subsec:accessing_code}

The code used for checking functions mentioned in the thesis is attached and can be accessed alongside this document.

\subsection{Key Files and Their Purposes}\label{subsec:key_files}

\begin{itemize}
    \item \nameref{sec:checkedfunctions.py}:
    This file houses the definitions of all functions used for checking integrals.
    \item \nameref{sec:symbols.py}:
    This file contains definitions of all symbols employed for computations with SymPy, a powerful library for symbolic mathematics.
\end{itemize}

\subsection{Working with functions}\label{subsec:working_with_funcs}

To obtain a function with symbols already incorporated, call it with an empty list of arguments (e.g., \mintinline{python}{function_name()}).

\subsection{Displaying Equations Effectively}\label{subsec:displaying_eqs}

\begin{enumerate}
    \item Import the \mintinline{python}{display} function from \mintinline{python}{IPython.display}.
    \item Use \mintinline{python}{display(..)} to present statements or even equations visually.
\end{enumerate}

\subsection{Handling Integrals}\label{subsec:handling_integrals}

\begin{enumerate}
    \item \emph{Displaying an Integral:}
    Use \mintinline{python}{sympy.Integral(..)} and put it into \mintinline{python}{display(..)} to visualie the integral.
    \item \emph{Computing an Integral Symbolically:}
    Apply \mintinline{python}{.doit(..)} to the integral object for symbolic computation.
    \item \emph{Approximating an Integral Numerically:}
    Add the option \mintinline{python}{"method=quad"} within the \mintinline{python}{.doit(..)} call to calculate the integral using the quadrature approximation method.
\end{enumerate}