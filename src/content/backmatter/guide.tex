\section{User Guide}\label{sec:user-guide}

\subsection{Accessing the code}\label{subsec:accessing-the-code}

The code used for checking functions mentioned in the thesis is attached
and can be accessed alongside this document.

\subsection{Key files and their purposes}\label{subsec:key-files-and-their-purposes}

\begin{itemize}
    \item \nameref{sec:checkedfunctions.py}:
    This file houses the definitions of all functions used for checking integrals.
    \item \nameref{sec:symbols.py}:
    This file contains definitions of all symbols employed for computations with SymPy,
    a powerful library for symbolic mathematics.
\end{itemize}

\subsection{Working with functions}\label{subsec:working-with-functions}

To obtain a function with symbols already incorporated,
call it with an empty list of arguments (e.g., \mintinline{python}{function_name()}).

\subsection{Displaying equations effectively}\label{subsec:displaying-equations-effectively}

\begin{enumerate}
    \item Import the \mintinline{python}{display} function from \mintinline{python}{IPython.display}.
    \item Use \mintinline{python}{display(..)} to present statements or even equations visually.
\end{enumerate}

\subsection{Handling integrals}\label{subsec:handling-integrals}

\begin{enumerate}
    \item \emph{Displaying an integral:}
    Use \mintinline{python}{sympy.Integral(..)}
    and put it into \mintinline{python}{display(..)} to visualize the integral.
    \item \emph{Computing an integral symbolically:}
    Apply \mintinline{python}{.doit(..)} to the integral object for symbolic computation.
    \item \emph{Approximating an integral numerically:}
    Add the option \mintinline{python}{"method=quad"} within the \mintinline{python}{.doit(..)} call
    to calculate the integral using the quadrature approximation method.
\end{enumerate}